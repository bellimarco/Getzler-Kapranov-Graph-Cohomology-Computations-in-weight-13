
\newcommand{\myB}{\bGK^{12,1}}

\section{Low excess computations in weight 13}\label{sec:low excess13}

\subsection{Blown-up pictures of graphs}\label{sec:blownup}
Generators of $\myB_{g,n}$ are connected graphs, with exactly one vertex of weight 11 or 13, and for the decoration of this vertex we may use the graphical notation of the previous section.
Here we perform explicit computations of the cohomology of $\myB_{g,n}$ for the pairs $(g,n)$ such that $3g + 2n$ is at most 27. In particular, we prove Proposition~\ref{prop:wt 13 vanishing} and Theorem~\ref{thm:lowexc13}. To this end, we need to enumerate all contributing graphs, and this is notationally easier if we draw our graphs differently, using the blown-up picture as in \cite{PayneWillwacher21, PayneWillwacher24}.
More precisely, for a weight 11 vertex decorated by $\omega_A$, we remove the genus $1$ vertex, leaving the edges incident to $v$ as external legs. Of those, we mark the ones in $A$ with a symbol $\omega$ and the others by $\epsilon$. 
\[
\begin{tikzpicture}
    \node[ext] (v) at (0,0) {};
    \node at (.5, .1) {$\scriptscriptstyle \vdots$};
    \draw (v) edge +(-.5,-.5) edge +(-.5,0) edge +(-.5,+.5) 
    edge[->-] +(.5,+.5) edge[->-] +(.5,+.3) edge[->-] +(.5,-.3) edge[->-] +(.5,-.5)
    ;
    \draw [decorate,decoration={brace,amplitude=5pt}]
  (.8,.5) -- (.8,-.5) node[midway,xshift=1em]{$A$};
  \end{tikzpicture}
\quad \to \quad
\begin{tikzpicture}
\node (e1) at (0,0) {$\epsilon$};
\node  at (.6,0) {$\cdots$};
\node (e2) at (1.2,0) {$\epsilon$};
\node (o1) at (1.7,0) {$\omega$};
\node  at (2.3,0) {$\cdots$};
\node (o2) at (2.9,0) {$\omega$};
\draw (e1) edge +(0,.5) (e2) edge +(0,.5) (o1) edge +(0,.5) (o2) edge +(0,.5) ;
\end{tikzpicture}
\]
Similarly, a weight $13$ vertex decorated by $Z_{B\subset A}$ is represented as in the following picture 
\[
\begin{tikzpicture}
    \node[ext] (v) at (0,0) {};
    \node[int] (w) at (0.7,0) {};
    \draw (v) edge +(-.5,.5) edge +(-.5,.3) edge[->-] +(-.5,.1) edge[crossed] (w) 
    (w) edge +(.5,-.5) edge +(.5,0) edge +(.5,.5);
    \draw (v) edge[->-] +(-.5,-.3) edge[->-] +(-.5,-.5);
    \draw [decorate,decoration={brace,amplitude=5pt}]
  (-1.2,-.5) -- (-1.2,.5) node[midway,xshift=-1em]{$A$};
    \draw [decorate,decoration={brace,amplitude=5pt}]
  (-.7,-.5) -- (-.7,.1) node[midway,xshift=-.7em]{$\scriptstyle B$};
\end{tikzpicture}
\to 
\begin{tikzpicture}
\node (e1) at (0,0) {$\epsilon$};
\node  at (.6,0) {$\cdots$};
\node (e2) at (1.2,0) {$\epsilon$};
\node (o1) at (1.7,0) {$\omega$};
\node  at (2.3,0) {$\cdots$};
\node (o2) at (2.9,0) {$\omega$};
\node (o3) at (3.4,0) {$\omega$};
\node[int] (i) at (3.4,.7) {};
\draw (e1) edge +(0,.5) (e2) edge +(0,.5) (o1) edge +(0,.5) (o2) edge +(0,.5) ;
\draw (i) edge[crossed] (o3) edge +(0,.5) edge +(0.5,.5) edge +(-0.5,.5);
\draw [decorate,decoration={brace,amplitude=5pt}]
  (3.1,-.3) -- (1.5,-.3) node[midway,yshift=-.9em]{$\scriptstyle B$};
\draw [decorate,decoration={brace,amplitude=5pt}]
  (3.1,-.9) -- (-.2,-.9) node[midway,yshift=-.9em]{$\scriptstyle A$};
\end{tikzpicture}
\]
We emphasize that this is merely a different way of drawing graphs.
Furthermore, we remind the reader that due to the relations on the decorations, our graph complex $\myB_{g,n}$ is not fully combinatorial, i.e., there are relations between some graphs.

\subsection{Excess}
We define the excess
\[
E(g,n) := 3g +2n - 25.
\]
We will compute the cohomology of $\myB_{g,n}$ for all pairs $g,n$ such that $E(g,n)\leq 2$. 

To this end, let us write each generator $\Gamma$ of $\myB_{g,n}$ as a union of its blown-up components:
\[
\Gamma = C_1\cup\cdots \cup C_k.
\]
Let $g_i$ be the contribution of $C_i$ to the genus of $\Gamma$.  More precisely,
\[
g_i = h^1(C_i) + \# \epsilon + \#\omega-1,
\]
i.e., the loop order of $C_i$ plus the number of its $\epsilon$ and $\omega$ labeled legs minus one.  Then we define the excess of $C_i$ as 
\begin{equation}\label{equ:exc comp}
E(C_i)
:= 3g_i +2n_i - 2\# \omega
=
3h^1(C_i) + 3\# \epsilon + \#\omega+2n_i -3. 
\end{equation}

\begin{lem}\label{lem:excess13}
The excess is additive over blown-up components, in the sense that for any graph $\Gamma=C_1\cup\cdots \cup C_k\in \myB_{g,n}$ we have
\begin{equation}\label{equ:E additive13}
E(g,n) = E(C_1)+ \cdots + E(C_k),  
\end{equation}
and the excess of each blown-up component is non-negative.  
\end{lem}
\begin{proof}
For the first statement note that $g= 1+\sum_{i=1}^kg_i$, and the total number of $\omega$-legs must be 11.
The second statement follows as in \cite[Lemma 4.2]{PayneWillwacher24}.
\end{proof}

The excess provides a measure of the combinatorial complexity of the relevant graphs. In particular, the additivity property of the above lemma helps us
to organize our calculations. By enumerating all blown up components of low excess, it becomes possible to compute $H^k(\myB_{g,n})$ by hand for all $g, n$ with $E(g, n) \leq 2$.

\subsection{Excess 0 and proof of Proposition \ref{prop:wt 13 vanishing}}
Blown-up components of excess 0 must have loop order 0 and at most 3 legs by \eqref{equ:exc comp} and are hence easily seen to be one of the following forms:
\[
  \begin{tikzpicture}
    \node (v) at (0,0) {$\omega$};
    \node (w) at (1,0) {$j$};
    \draw (v) edge (w);
  \end{tikzpicture}
      \quad\quad
      \text{or} 
      \quad\quad
    \begin{tikzpicture}
        \node[int] (i) at (0,.5) {};
        \node (v1) at (-.5,-.2) {$\omega$};
        \node (v2) at (0,-.2) {$\omega$};
        \node (v3) at (.5,-.2) {$\omega$};
      \draw (i) edge (v1) edge (v2) edge (v3);
      \end{tikzpicture} 
      \quad\quad 
      \text{or} 
      \quad\quad 
      \begin{tikzpicture}
        \node[int] (i) at (0,.5) {};
        \node (v1) at (-.5,-.2) {$\omega$};
        \node (v2) at (0,-.2) {$\omega$};
        \node (v3) at (.5,-.2) {$\omega$};
      \draw (i) edge[crossed] (v1) edge (v2) edge (v3);
      \end{tikzpicture} 
      .
\]
Using the conventions introduced in \S \ref{pics},
each generator of $\myB_{g,n}$ must have exactly one marked edge. 
Graphs with the third component in the list above are however zero in $\myB_{g,n}$ by \eqref{tripodzero}.
This means that every generator $\Gamma\in \myB_{g,n}$ must have a blown-up component of excess at least one, and we arrive at the following corollary of Lemma \ref{lem:excess13}, that is also stated as Proposition \ref{prop:wt 13 vanishing} in the introduction.

\begin{cor} \label{cor:Eneg13}
If $E(g,n) <1$ then $
\gr^W_{13} H^*_c(\M_{g,n}) =0.$
\end{cor}


\subsection{Excess 1} \label{sec:exc1-13}
We next consider the case of excess one. In this case generators of $\myB_{g,n}$ must be unions of blown-up components of excess zero as above, and one blown-up component of excess one, containing a crossed edge.
The combinatorially possible excess 1 components with a crossed edge are:
\begin{equation}\label{equ:exc 1 special}
\begin{aligned}
    \begin{tikzpicture}
        \node[int] (i) at (0,.5) {};
        \node (v1) at (-.5,-.2) {$\omega$};
        \node (v2) at (0,-.2) {$\omega$};
        \node (v3) at (.5,-.2) {$j$};
      \draw (i) edge[crossed] (v1) edge (v2) edge (v3);
      \end{tikzpicture}
      =-\frac 1{24} \bigg(
            \begin{tikzpicture}
        \node (i) at (0,0) {$\omega$};
        \node (v2) at (.7,0) {$j$};
      \draw (i) edge (v2) ;
      \end{tikzpicture}
        \begin{tikzpicture}
        \node[int] (i) at (0,.5) {};
        \node (v2) at (0,-.2) {$\omega$};
      \draw (i) edge[crossed, loop] (i) edge (v2) ;
      \end{tikzpicture}
      \bigg)
      \quad\text{or}\quad 
      \begin{tikzpicture}
        \node[int] (i) at (0,.5) {};
        \node (v2) at (0,-.2) {$\omega$};
      \draw (i) edge[crossed, loop] (i) edge (v2) ;
      \end{tikzpicture}
      \quad\text{or}\quad 
      \begin{tikzpicture}
        \node[int] (i) at (0,.5) {};
        \node[int] (j) at (1,.5) {};
        \node (v1) at (-.5,-.2) {$\omega$};
        \node (v2) at (0,-.2) {$\omega$};
        \node (v3) at (1,-.2) {$\omega$};
        \node (v4) at (1.5,-.2) {$\omega$};
      \draw (i) edge (v1) edge (v2) edge[crossed] (j) (j) edge (v3) edge (v4);
      \end{tikzpicture}
      \\
      \quad\text{or}\quad 
      \begin{tikzpicture}
        \node[int] (i) at (0,.5) {};
        \node[int] (j) at (1,.5) {};
        \node (v1) at (-.5,-.2) {$\omega$};
        \node (v2) at (0,-.2) {$\omega$};
        \node (v3) at (1,-.2) {$\omega$};
        \node (v4) at (1.5,-.2) {$\omega$};
      \draw (i) edge[crossed] (v1) edge (v2) edge (j) (j) edge (v3) edge (v4);
      \end{tikzpicture}=-\frac1{24}\bigg(
      \begin{tikzpicture}
        \node[int] (i) at (0,.5) {};
        \node (v2) at (0,-.2) {$\omega$};
      \draw (i) edge[crossed, loop] (i) edge (v2) ;
      \end{tikzpicture}
        \begin{tikzpicture}
    \node[int] (i) at (0,.5) {};
    \node (v1) at (-.5,-.2) {$\omega$};
    \node (v2) at (0,-.2) {$\omega$};
    \node (v3) at (.5,-.2) {$\omega$};
  \draw (i) edge (v1) edge (v2) edge (v3);
  \end{tikzpicture}
      \bigg)
      \quad\text{or}\quad 
      \begin{tikzpicture}
        \node[int] (i) at (0,.5) {};
        \node (v1) at (-.7,-.2) {$\omega$};
        \node (v2) at (-.2,-.2) {$\omega$};
        \node (v3) at (.2,-.2) {$\omega$};
        \node (v4) at (.7,-.2) {$\omega$};
      \draw (i) edge[crossed] (v1) edge (v2) edge (v3) edge (v4);
      \end{tikzpicture}
      .
\end{aligned}
\end{equation}
The first and the fourth component can be replaced by other terms using the relations in Remark \ref{rem:graph vanishing}.
Thus, if $E(g,n) = 1$, we have three possible generators for $\myB_{g,n}$: 
\[
  \Gamma^{(0)}_{irr}=
  \begin{tikzpicture}
    \node (v) at (0,0) {$\omega$};
    \node (w) at (1,0) {$1$};
    \draw (v) edge (w);
  \end{tikzpicture}
  \cdots 
  \begin{tikzpicture}
    \node (v) at (0,0) {$\omega$};
    \node (w) at (1,0) {$n$};
    \draw (v) edge (w);
  \end{tikzpicture}
  \begin{tikzpicture}
    \node[int] (i) at (0,.5) {};
    \node (v1) at (-.5,-.2) {$\omega$};
    \node (v2) at (0,-.2) {$\omega$};
    \node (v3) at (.5,-.2) {$\omega$};
  \draw (i) edge (v1) edge (v2) edge (v3);
  \end{tikzpicture}
  \cdots 
  \begin{tikzpicture}
    \node[int] (i) at (0,.5) {};
    \node (v1) at (-.5,-.2) {$\omega$};
    \node (v2) at (0,-.2) {$\omega$};
    \node (v3) at (.5,-.2) {$\omega$};
  \draw (i) edge (v1) edge (v2) edge (v3);
  \end{tikzpicture}
  \begin{tikzpicture}
    \node[int] (i) at (0,.5) {};
    \node (v2) at (0,-.2) {$\omega$};
  \draw (i) edge[crossed, loop] (i) edge (v2) ;
  \end{tikzpicture}
\]
\[
  \Gamma^{(0)}_{\delta}=
  \begin{tikzpicture}
    \node (v) at (0,0) {$\omega$};
    \node (w) at (1,0) {$1$};
    \draw (v) edge (w);
  \end{tikzpicture}
  \cdots 
  \begin{tikzpicture}
    \node (v) at (0,0) {$\omega$};
    \node (w) at (1,0) {$n$};
    \draw (v) edge (w);
  \end{tikzpicture}
  \begin{tikzpicture}
    \node[int] (i) at (0,.5) {};
    \node (v1) at (-.5,-.2) {$\omega$};
    \node (v2) at (0,-.2) {$\omega$};
    \node (v3) at (.5,-.2) {$\omega$};
  \draw (i) edge (v1) edge (v2) edge (v3);
  \end{tikzpicture}
  \cdots 
  \begin{tikzpicture}
    \node[int] (i) at (0,.5) {};
    \node (v1) at (-.5,-.2) {$\omega$};
    \node (v2) at (0,-.2) {$\omega$};
    \node (v3) at (.5,-.2) {$\omega$};
  \draw (i) edge (v1) edge (v2) edge (v3);
  \end{tikzpicture}
  \begin{tikzpicture}
    \node[int] (i) at (0,.5) {};
    \node[int] (j) at (1,.5) {};
    \node (v1) at (-.5,-.2) {$\omega$};
    \node (v2) at (0,-.2) {$\omega$};
    \node (v3) at (1,-.2) {$\omega$};
    \node (v4) at (1.5,-.2) {$\omega$};
  \draw (i) edge (v1) edge (v2) edge[crossed] (j) (j) edge (v3) edge (v4);
  \end{tikzpicture}
\]
\[
  \Gamma^{(0)}_{s}=
  \begin{tikzpicture}
    \node (v) at (0,0) {$\omega$};
    \node (w) at (1,0) {$1$};
    \draw (v) edge (w);
  \end{tikzpicture}
  \cdots 
  \begin{tikzpicture}
    \node (v) at (0,0) {$\omega$};
    \node (w) at (1,0) {$n$};
    \draw (v) edge (w);
  \end{tikzpicture}
  \begin{tikzpicture}
    \node[int] (i) at (0,.5) {};
    \node (v1) at (-.5,-.2) {$\omega$};
    \node (v2) at (0,-.2) {$\omega$};
    \node (v3) at (.5,-.2) {$\omega$};
  \draw (i) edge (v1) edge (v2) edge (v3);
  \end{tikzpicture}
  \cdots 
  \begin{tikzpicture}
    \node[int] (i) at (0,.5) {};
    \node (v1) at (-.5,-.2) {$\omega$};
    \node (v2) at (0,-.2) {$\omega$};
    \node (v3) at (.5,-.2) {$\omega$};
  \draw (i) edge (v1) edge (v2) edge (v3);
  \end{tikzpicture}
  \begin{tikzpicture}
    \node[int] (i) at (0,.5) {};
    \node (v1) at (-.7,-.2) {$\omega$};
    \node (v2) at (-.2,-.2) {$\omega$};
    \node (v3) at (.2,-.2) {$\omega$};
    \node (v4) at (.7,-.2) {$\omega$};
  \draw (i) edge[crossed] (v1) edge (v2) edge (v3) edge (v4);
  \end{tikzpicture}
\]
In each case, the total number of $\omega$-legs must be 11.
Note that the graphs $\Gamma^{(0)}_{\delta}$ and $\Gamma^{(0)}_{s}$ only exist in genera $g\geq 4$. When $g\geq 4$ and $n$ is such that $3g+2n=26$, the three graphs above form a basis of $\myB_{g,n}$. In genus $2$, we have that $\myB_{2,10}$ is one-dimensional, spanned by $\Gamma^{(0)}_{irr}$.

The cohomological degree of the graphs $\Gamma^{(0)}_{irr}$ and $\Gamma^{(0)}_{\delta}$ above is $k=24-n$, while that of $\Gamma^{(0)}_{s}$ is $k=23-n$.
Using \eqref{equ:split 13},
one checks that the differential sends $\Gamma^{(0)}_{s}$ to a non-zero multiple of $\Gamma^{(0)}_{\delta}$ plus a multiple of $\Gamma^{(0)}_{irr}$. Note that of the six terms in \eqref{equ:split 13} only the the third and the fifth term contribute, with the third term producing a multiple of $\Gamma^{(0)}_{irr}$ (after using the relation \eqref{rel1}), and the fifth term producing $-\Gamma^{(0)}_{\delta}$ (after replacing the $\psi$ class with a boundary class). In particular, the first term in the second line of \eqref{equ:split 13} is zero in this case because there cannot be a weight 13 vertex of genus 1 and valence 11.
In the end, we retain only one-dimensional cohomology, generated by $\Gamma^{(0)}_{irr}$. We hence obtain:
\begin{align*}   H^k(\myB_{2,10}) &=
    \begin{cases}
        V_{1^{10}} & \text{for $k=14$} \\
        0 & \text{otherwise}
    \end{cases}
&
H^k(\myB_{4,7}) &=
    \begin{cases}
        % V_{1^{7}}\oplus 
        V_{1^{7}} & \text{for $k=17$} \\
        0 & \text{otherwise}
    \end{cases}
\\
H^k(\myB_{6,4}) &=
\begin{cases}
    % V_{1^4} \oplus 
    V_{1^{4}} & \text{for $k=20$} \\
    0 & \text{otherwise}
\end{cases}     
&
H^k(\myB_{8,1}) &=
\begin{cases}
    % V_{1} \oplus 
    V_1 & \text{for $k=23$} \\
    0 & \text{otherwise.}
\end{cases}  
\end{align*}



\subsection{Excess 2} \label{sec:exc2-13}
A general generator $\Gamma$ of $\myB_{g,n}$ in excess 2 can have either two blown-up components of excess one each, or one component of excess 2, necessarily with a crossed edge.
As in \cite[Section 4.2]{PayneWillwacher24}, the relevant excess one components without crossed edge are 
\begin{equation}\label{equ:exc 1 trees}
  \begin{tikzpicture}
    \node (v) at (0,0) {$\omega$};
    \node (w) at (1,0) {$\epsilon$};
    \draw (v) edge (w);
  \end{tikzpicture}
\quad\text{or}\quad  
\begin{tikzpicture}
  \node[int] (i) at (0,.5) {};
  \node (v1) at (-.5,-.2) {$j$};
  \node (v2) at (0,-.2) {$\omega$};
  \node (v3) at (.5,-.2) {$\omega$};
\draw (i) edge (v1) edge (v2) edge (v3);
\end{tikzpicture}
\quad\text{or}\quad  
\begin{tikzpicture}
  \node[int] (i) at (0,.5) {};
  \node (v1) at (-.6,-.2) {$\omega$};
  \node (v2) at (-.2,-.2) {$\omega$};
  \node (v3) at (.2,-.2) {$\omega$};
  \node (v4) at (.6,-.2) {$\omega$};
\draw (i) edge (v1) edge (v2) edge (v3) edge (v4);
\end{tikzpicture}
\quad\text{or}\quad  
\begin{tikzpicture}
  \node[int] (i) at (0,.5) {};
  \node[int] (j) at (1,.5) {};
  \node (v1) at (-.5,-.2) {$\omega$};
  \node (v2) at (0,-.2) {$\omega$};
  \node (v3) at (1,-.2) {$\omega$};
  \node (v4) at (1.5,-.2) {$\omega$};
\draw (i) edge (v1) edge (v2) edge (j) (j) edge (v3) edge (v4);
\end{tikzpicture}
\, .
\end{equation}
Combining one such blown-up component and one from \eqref{equ:exc 1 special} we obtain 12 possible generators with two blown-up components of excess 1.
There is a larger set of possible excess 2 components containing a crossed edge, listed below.
\begin{align*}
 & \begin{tikzpicture}
    \node[int] (i) at (0,.5) {};
    \node[int] (j) at (1,.5) {};
    % \node (v1) at (-.5,-.2) {$\omega$};
    \node (v2) at (0,-.2) {$\omega$};
    \node (v3) at (1,-.2) {$\omega$};
    % \node (v4) at (1.5,-.2) {$\omega$};
  \draw (i) edge[bend left] (j) edge (v2) edge[crossed] (j) (j) edge (v3);
  \end{tikzpicture}
  =
    \begin{tikzpicture}
      \node[int] (i) at (0,.5) {};
      \node[int] (j) at (1,.5) {};
      \node (v1) at (-.5,-.2) {$\omega$};
      \node (v2) at (0,-.2) {$\omega$};
    \draw (i) edge (v1) edge (v2) edge[crossed] (j) (j) edge[loop] (j);
    \end{tikzpicture}
    =(\cdots)
  \quad\text{or}\quad 
  \begin{tikzpicture}
    \node[int] (i) at (0,.5) {};
    \node (v1) at (-.5,-.2) {$\omega$};
    \node (v2) at (0,-.2) {$i$};
    \node (v3) at (.5,-.2) {$j$};
  \draw (i) edge[crossed] (v1) edge (v2) edge (v3);
  \end{tikzpicture}
  \quad\text{or}\quad 
  \begin{tikzpicture}
    \node[int] (i) at (0,.5) {};
    \node[int] (j) at (1,.5) {};
    \node (v1) at (-.5,-.2) {$\omega$};
    \node (v2) at (0,-.2) {$\omega$};
    \node (v3) at (1,-.2) {$\omega$};
    \node (v4) at (1.5,-.2) {$i$};
  \draw (i) edge[crossed] (v1) edge (v2) edge (j) (j) edge (v3) edge (v4);
  \end{tikzpicture}
  =(\cdots)
  \\&
  \quad\text{or}\quad 
  \begin{tikzpicture}
    \node[int] (i) at (0,.5) {};
    \node (v1) at (-.5,-.2) {$\omega$};
    \node (v2) at (0,-.2) {$\omega$};
    \node (v3) at (.5,-.2) {$\epsilon$};
  \draw (i) edge[crossed] (v1) edge (v2) edge (v3);
  \end{tikzpicture}
  =(\cdots)
  \quad\text{or}\quad 
  \begin{tikzpicture}
    \node[int] (i) at (0,.5) {};
    \node[int] (j) at (1,.5) {};
    \node (v1) at (-.5,-.2) {$\omega$};
    \node (v2) at (0,-.2) {$i$};
    \node (v3) at (1,-.2) {$\omega$};
    \node (v4) at (1.5,-.2) {$\omega$};
  \draw (i) edge[crossed] (v1) edge (v2) edge (j) (j) edge (v3) edge (v4);
  \end{tikzpicture}
  \quad\text{or}\quad 
  \begin{tikzpicture}
    \node[int] (i) at (0,.5) {};
    \node (v1) at (-.7,-.2) {$\omega$};
    \node (v2) at (-.2,-.2) {$\omega$};
    \node (v3) at (.2,-.2) {$\omega$};
    \node (v4) at (.7,-.2) {$i$};
  \draw (i) edge[crossed] (v1) edge (v2) edge (v3) edge (v4);
  \end{tikzpicture}
  \quad\text{or}\quad  %\mapsto 
  \begin{tikzpicture}
    \node[int] (i) at (0,.5) {};
    \node[int] (j) at (1,.5) {};
    \node (v1) at (-.5,-.2) {$\omega$};
    \node (v2) at (0,-.2) {$\omega$};
    \node (v3) at (1,-.2) {$\omega$};
    \node (v4) at (1.5,-.2) {$i$};
  \draw (i) edge (v1) edge (v2) edge[crossed] (j) (j) edge (v3) edge (v4);
  \end{tikzpicture}
  \\&
  \quad\text{or}\quad 
  \begin{tikzpicture}
    \node[int] (i) at (0,.5) {};
    \node (v1) at (-.3,-.2) {$\omega$};
    \node (v2) at (.3,-.2) {$\omega$};
  \draw (i) edge[crossed, loop] (i) edge (v2) edge (v1);
  \end{tikzpicture}
  =0
  \quad\text{or}\quad
  \begin{tikzpicture}
    \node[int] (i) at (0,.5) {};
    \node[int] (j) at (0,1.2) {};
    \node (v1) at (-.3,-.2) {$\omega$};
    \node (v2) at (.3,-.2) {$\omega$};
  \draw (j) edge[crossed, loop] (j) edge (i) (i) edge (v2) edge (v1);
  \end{tikzpicture}=0
  \quad\text{or}\quad 
  \begin{tikzpicture}
    \node[int] (i) at (0,.5) {};
    \node[int] (j) at (1,.5) {};
    \node (v1) at (-.5,-.2) {$\omega$};
    \node (v2) at (0,-.2) {$\omega$};
    \node (v3) at (1,-.2) {$\omega$};
    \node (v4) at (1.5,-.2) {$\omega$};
    \node (v5) at (.5,-.2) {$\omega$};
  \draw (i) edge (v1) edge (v2) edge (v5) edge[crossed] (j) (j) edge (v3) edge (v4);
  \end{tikzpicture}
  \quad\text{or}\quad %\mapsto
  \begin{tikzpicture}
    \node[int] (i) at (0,.5) {};
    \node[int] (j) at (1,.5) {};
    \node[int] (k) at (-1,.5) {};
    \node (v1) at (-1.5,-.2) {$\omega$};
    \node (v2) at (0,-.2) {$\omega$};
    \node (v3) at (1,-.2) {$\omega$};
    \node (v4) at (1.5,-.2) {$\omega$};
    \node (v5) at (-1,-.2) {$\omega$};
  \draw (i) edge (k) edge (v2)  edge[crossed] (j) (j) edge (v3) edge (v4)
  (k) edge (v5) edge (v1);
  \end{tikzpicture}
  \\&
  \quad\text{or}\quad 
  \begin{tikzpicture}
    \node[int] (i) at (0,.5) {};
    \node[int] (j) at (1,.5) {};
    \node (v1) at (-.5,-.2) {$\omega$};
    \node (v2) at (0,-.2) {$\omega$};
    \node (v3) at (1,-.2) {$\omega$};
    \node (v4) at (1.5,-.2) {$\omega$};
    \node (v5) at (.5,-.2) {$\omega$};
  \draw (i) edge[crossed] (v1) edge (v2) edge (j) (j) edge (v3) edge (v4) edge (v5);
  \end{tikzpicture}=(\cdots)
  \quad\text{or}\quad
  \\& 
  \begin{tikzpicture}
    \node[int] (i) at (0,.5) {};
    \node (v1) at (-.7,-.2) {$\omega$};
    \node (v2) at (-.2,-.2) {$\omega$};
    \node (v3) at (.2,-.2) {$\omega$};
    \node (v4) at (.7,-.2) {$\omega$};
    \node (v5) at (1.2,-.2) {$\omega$};
  \draw (i) edge[crossed] (v1) edge (v2) edge (v3) edge (v4) edge (v5);
  \end{tikzpicture}
  \quad\text{or}\quad %\mapsto
  \begin{tikzpicture}
    \node[int] (i) at (0,.5) {};
    \node[int] (j) at (1,.5) {};
    \node (v1) at (-.5,-.2) {$\omega$};
    \node (v2) at (0,-.2) {$\omega$};
    \node (v3) at (1,-.2) {$\omega$};
    \node (v4) at (1.5,-.2) {$\omega$};
    \node (v5) at (.5,-.2) {$\omega$};
  \draw (i) edge[crossed] (v1) edge (v2) edge (v5) edge (j) (j) edge (v3) edge (v4);
  \end{tikzpicture}
  \\&
  \begin{tikzpicture}
    \node[int] (i) at (0,.5) {};
    \node[int] (j) at (1,.5) {};
    \node[int] (k) at (-1,.5) {};
    \node (v1) at (-1.5,-.2) {$\omega$};
    \node (v2) at (0,-.2) {$\omega$};
    \node (v3) at (1,-.2) {$\omega$};
    \node (v4) at (1.5,-.2) {$\omega$};
    \node (v5) at (-1,-.2) {$\omega$};
  \draw (i) edge (k) edge (v2)  edge (j) (j) edge (v3) edge (v4)
  (k) edge (v5) edge[crossed] (v1);
  \end{tikzpicture}=(\cdots)
  \quad\text{or}\quad
  \begin{tikzpicture}
    \node[int] (i) at (0,.5) {};
    \node[int] (j) at (1,.5) {};
    \node[int] (k) at (-1,.5) {};
    \node (v1) at (-1.5,-.2) {$\omega$};
    \node (v2) at (0,-.2) {$\omega$};
    \node (v3) at (1,-.2) {$\omega$};
    \node (v4) at (1.5,-.2) {$\omega$};
    \node (v5) at (-1,-.2) {$\omega$};
  \draw (i) edge (k) edge[crossed] (v2)  edge (j) (j) edge (v3) edge (v4)
  (k) edge (v5) edge (v1);
  \end{tikzpicture}
  \substack{\text{symm.} \\ =}0
\end{align*}
The first two graphs in the third line are zero in $\bGK_{g,n}^{12,1}$ because of rule (3) above. Additionally, some graphs can be expressed through other generators via the relations, which is written as $=(\cdots)$, see again Remark \ref{rem:graph vanishing}. 
We hence end up with an additional 8 generators with one blown-up component of excess 2.
Collecting all 20 generators, the differential is as follows:

\smallskip

\underline{Degree $25-n$:}
\begin{align*}
\Gamma_{i}^{(2)}:=
\scalebox{.6}{$\omis\tripods  
  \begin{tikzpicture}
    \node[int] (i) at (0,.5) {};
    \node[int] (j) at (1,.5) {};
    \node (v1) at (-.5,-.2) {$\omega$};
    \node (v2) at (0,-.2) {$i$};
    \node (v3) at (1,-.2) {$\omega$};
    \node (v4) at (1.5,-.2) {$\omega$};
  \draw (i) edge[crossed] (v1) edge (v2) edge (j) (j) edge (v3) edge (v4);
  \end{tikzpicture} $}
  &\mapsto 0
  \\
  \Gamma_{ij}^{(2)}:=
\scalebox{.6}{$\omis
\tripods
\begin{tikzpicture}
    \node[int] (i) at (0,.5) {};
    \node (v1) at (-.5,-.2) {$\omega$};
    \node (v2) at (0,-.2) {$i$};
    \node (v3) at (.5,-.2) {$j$};
  \draw (i) edge[crossed] (v1) edge (v2) edge (v3);
  \end{tikzpicture}
  $}
  &\mapsto 0
  \\
\Gamma^{(2)}_{j,irr}:=
\scalebox{.6}{$
  \begin{tikzpicture}
    \node (v) at (0,0) {$\omega$};
    \node (w) at (1,0) {$1$};
    \draw (v) edge (w);
  \end{tikzpicture}
  \cdots 
  \begin{tikzpicture}
    \node (v) at (0,0) {$\omega$};
    \node (w) at (1,0) {$n$};
    \draw (v) edge (w);
  \end{tikzpicture}
  \begin{tikzpicture}
    \node[int] (i) at (0,.5) {};
    \node (v1) at (-.5,-.2) {$\omega$};
    \node (v2) at (0,-.2) {$\omega$};
    \node (v3) at (.5,-.2) {$\omega$};
  \draw (i) edge (v1) edge (v2) edge (v3);
  \end{tikzpicture}
  \cdots 
  \begin{tikzpicture}
    \node[int] (i) at (0,.5) {};
    \node (v1) at (-.5,-.2) {$\omega$};
    \node (v2) at (0,-.2) {$\omega$};
    \node (v3) at (.5,-.2) {$\omega$};
  \draw (i) edge (v1) edge (v2) edge (v3);
  \end{tikzpicture}
  \begin{tikzpicture}
    \node[int] (i) at (0,.5) {};
    \node (v1) at (-.5,-.2) {$j$};
    \node (v2) at (0,-.2) {$\omega$};
    \node (v3) at (.5,-.2) {$\omega$};
  \draw (i) edge (v1) edge (v2) edge (v3);
  \end{tikzpicture}
  \begin{tikzpicture}
    \node[int] (i) at (0,.5) {};
    \node (v2) at (0,-.2) {$\omega$};
  \draw (i) edge[crossed, loop] (i) edge (v2) ;
  \end{tikzpicture}
  $}
  &\mapsto 0
\\
\Gamma_{birr}^{(2)}:=
\scalebox{.6}{$\omis\tripods 
\begin{tikzpicture}
  \node[int] (i) at (0,.5) {};
  \node[int] (j) at (1,.5) {};
  \node (v1) at (-.5,-.2) {$\omega$};
  \node (v2) at (0,-.2) {$\omega$};
  \node (v3) at (1,-.2) {$\omega$};
  \node (v4) at (1.5,-.2) {$\omega$};
\draw (i) edge (v1) edge (v2) edge (j) (j) edge (v3) edge (v4);
\end{tikzpicture}
\begin{tikzpicture}
        \node[int] (i) at (0,.5) {};
        \node (v2) at (0,-.2) {$\omega$};
      \draw (i) edge[crossed, loop] (i) edge (v2) ;
\end{tikzpicture}
$}&\mapsto 0
\\
\Gamma_{b\bar b}^{(2)}:=
\scalebox{.6}{$\omis\tripods 
\begin{tikzpicture}
  \node[int] (i) at (0,.5) {};
  \node[int] (j) at (1,.5) {};
  \node (v1) at (-.5,-.2) {$\omega$};
  \node (v2) at (0,-.2) {$\omega$};
  \node (v3) at (1,-.2) {$\omega$};
  \node (v4) at (1.5,-.2) {$\omega$};
\draw (i) edge (v1) edge (v2) edge (j) (j) edge (v3) edge (v4);
\end{tikzpicture}
\begin{tikzpicture}
        \node[int] (i) at (0,.5) {};
        \node[int] (j) at (1,.5) {};
        \node (v1) at (-.5,-.2) {$\omega$};
        \node (v2) at (0,-.2) {$\omega$};
        \node (v3) at (1,-.2) {$\omega$};
        \node (v4) at (1.5,-.2) {$\omega$};
      \draw (i) edge (v1) edge (v2) edge[crossed] (j) (j) edge (v3) edge (v4);
      \end{tikzpicture}
$}
&\mapsto 0
\\
\Gamma_{j\bar b}^{(2)}:=
\scalebox{.6}{$\omis\tripods 
\begin{tikzpicture}
  \node[int] (i) at (0,.5) {};
  \node (v1) at (-.5,-.2) {$j$};
  \node (v2) at (0,-.2) {$\omega$};
  \node (v3) at (.5,-.2) {$\omega$};
\draw (i) edge (v1) edge (v2) edge (v3);
\end{tikzpicture}
  \begin{tikzpicture}
    \node[int] (i) at (0,.5) {};
    \node[int] (j) at (1,.5) {};
    \node (v1) at (-.5,-.2) {$\omega$};
    \node (v2) at (0,-.2) {$\omega$};
    \node (v3) at (1,-.2) {$\omega$};
    \node (v4) at (1.5,-.2) {$\omega$};
  \draw (i) edge (v1) edge (v2) edge[crossed] (j) (j) edge (v3) edge (v4);
  \end{tikzpicture}
$}
&\mapsto 0
\\
  \Gamma_{\bar bi}^{(2)}:=
\scalebox{.6}{$\omis\tripods 
  \begin{tikzpicture}
    \node[int] (i) at (0,.5) {};
    \node[int] (j) at (1,.5) {};
    \node (v1) at (-.5,-.2) {$\omega$};
    \node (v2) at (0,-.2) {$\omega$};
    \node (v3) at (1,-.2) {$\omega$};
    \node (v4) at (1.5,-.2) {$i$};
  \draw (i) edge (v1) edge (v2) edge[crossed] (j) (j) edge (v3) edge (v4);
  \end{tikzpicture}$}
  &\mapsto 0
  \\
  \Gamma_{\bar b\bar b}^{(2)}:= \scalebox{.6}{$\omis\tripods  
  \begin{tikzpicture}
    \node[int] (i) at (0,.5) {};
    \node[int] (j) at (1,.5) {};
    \node[int] (k) at (-1,.5) {};
    \node (v1) at (-1.5,-.2) {$\omega$};
    \node (v2) at (0,-.2) {$\omega$};
    \node (v3) at (1,-.2) {$\omega$};
    \node (v4) at (1.5,-.2) {$\omega$};
    \node (v5) at (-1,-.2) {$\omega$};
  \draw (i) edge (k) edge (v2)  edge[crossed] (j) (j) edge (v3) edge (v4)
  (k) edge (v5) edge (v1);
  \end{tikzpicture}$}
  &\mapsto 0
\end{align*}

Note that the generators $\Gamma_{ij}^{(2)}$, $\Gamma_{i}^{(2)}$ are not independent; there are relations from the weight 13 relations, depending on the genus.
If $g=1$, only $\Gamma_{ij}^{(2)}$ contributes, and modulo relations the contribution is the irreducible representation $V_{21^{n-2}}$.
If $g\geq 3$ and $n\geq 2$, the $\Gamma^{(2)}_{i}$ are linearly independent, the $\Gamma_{ij}^{(2)}$ can be expressed through the $\Gamma^{(2)}_{i}$, and the overall $\ss_n$-representation is $V_{1^n} \oplus V_{21^{n-2}}$.

\smallskip

\underline{Degree $24-n$:}
\begin{align*}
\Gamma_{4irr}^{(2)}:= \scalebox{.6}{$\omis\tripods 
\begin{tikzpicture}
  \node[int] (i) at (0,.5) {};
  \node (v1) at (-.6,-.2) {$\omega$};
  \node (v2) at (-.2,-.2) {$\omega$};
  \node (v3) at (.2,-.2) {$\omega$};
  \node (v4) at (.6,-.2) {$\omega$};
\draw (i) edge (v1) edge (v2) edge (v3) edge (v4);
\end{tikzpicture}
\begin{tikzpicture}
        \node[int] (i) at (0,.5) {};
        \node (v2) at (0,-.2) {$\omega$};
      \draw (i) edge[crossed, loop] (i) edge (v2) ;
\end{tikzpicture}
$}
&\mapsto 3
\Gamma_{birr}^{(2)}
\\
\Gamma_{4\bar b}^{(2)}:=\scalebox{.6}{$\omis\tripods 
\begin{tikzpicture}
  \node[int] (i) at (0,.5) {};
  \node (v1) at (-.6,-.2) {$\omega$};
  \node (v2) at (-.2,-.2) {$\omega$};
  \node (v3) at (.2,-.2) {$\omega$};
  \node (v4) at (.6,-.2) {$\omega$};
\draw (i) edge (v1) edge (v2) edge (v3) edge (v4);
\end{tikzpicture}
\begin{tikzpicture}
        \node[int] (i) at (0,.5) {};
        \node[int] (j) at (1,.5) {};
        \node (v1) at (-.5,-.2) {$\omega$};
        \node (v2) at (0,-.2) {$\omega$};
        \node (v3) at (1,-.2) {$\omega$};
        \node (v4) at (1.5,-.2) {$\omega$};
      \draw (i) edge (v1) edge (v2) edge[crossed] (j) (j) edge (v3) edge (v4);
      \end{tikzpicture}
$}
&\mapsto 3
\Gamma_{b\bar b}^{(2)}
\\
\Gamma_{j\bar 4}^{(2)}:=\scalebox{.6}{$\omis\tripods 
\begin{tikzpicture}
  \node[int] (i) at (0,.5) {};
  \node (v1) at (-.5,-.2) {$j$};
  \node (v2) at (0,-.2) {$\omega$};
  \node (v3) at (.5,-.2) {$\omega$};
\draw (i) edge (v1) edge (v2) edge (v3);
\end{tikzpicture}
  \begin{tikzpicture}
    \node[int] (i) at (0,.5) {};
    \node (v1) at (-.7,-.2) {$\omega$};
    \node (v2) at (-.2,-.2) {$\omega$};
    \node (v3) at (.2,-.2) {$\omega$};
    \node (v4) at (.7,-.2) {$\omega$};
  \draw (i) edge[crossed] (v1) edge (v2) edge (v3) edge (v4);
  \end{tikzpicture}
$}
&\mapsto -
\Gamma_{j\bar b}^{(2)}+ (const) \Gamma_{j,irr}^{(2)}
\\
\Gamma^{(2)}_{\epsilon,irr}:=
\scalebox{.6}{$
  \begin{tikzpicture}
    \node (v) at (0,0) {$\omega$};
    \node (w) at (1,0) {$1$};
    \draw (v) edge (w);
  \end{tikzpicture}
  \cdots 
  \begin{tikzpicture}
    \node (v) at (0,0) {$\omega$};
    \node (w) at (1,0) {$n$};
    \draw (v) edge (w);
  \end{tikzpicture}
  \begin{tikzpicture}
    \node[int] (i) at (0,.5) {};
    \node (v1) at (-.5,-.2) {$\omega$};
    \node (v2) at (0,-.2) {$\omega$};
    \node (v3) at (.5,-.2) {$\omega$};
  \draw (i) edge (v1) edge (v2) edge (v3);
  \end{tikzpicture}
  \cdots 
  \begin{tikzpicture}
    \node[int] (i) at (0,.5) {};
    \node (v1) at (-.5,-.2) {$\omega$};
    \node (v2) at (0,-.2) {$\omega$};
    \node (v3) at (.5,-.2) {$\omega$};
  \draw (i) edge (v1) edge (v2) edge (v3);
  \end{tikzpicture}
  \begin{tikzpicture}
    \node (v) at (0,0) {$\omega$};
    \node (w) at (1,0) {$\epsilon$};
    \draw (v) edge (w);
  \end{tikzpicture}
  \begin{tikzpicture}
    \node[int] (i) at (0,.5) {};
    \node (v2) at (0,-.2) {$\omega$};
  \draw (i) edge[crossed, loop] (i) edge (v2) ;
  \end{tikzpicture}
  $}
  &\mapsto  \sum_j \pm \Gamma_{j,irr}^{(2)} +(const) \Gamma_{birr}^{(2)}
\\
\Gamma_{\bar B}^{(2)}:=
\scalebox{.6}{$\omis
\tripods
\begin{tikzpicture}
    \node[int] (i) at (0,.5) {};
    \node[int] (j) at (1,.5) {};
    \node (v1) at (-.5,-.2) {$\omega$};
    \node (v2) at (0,-.2) {$\omega$};
    \node (v3) at (1,-.2) {$\omega$};
    \node (v4) at (1.5,-.2) {$\omega$};
    \node (v5) at (.5,-.2) {$\omega$};
  \draw (i) edge (v1) edge (v2) edge (v5) edge[crossed] (j) (j) edge (v3) edge (v4);
  \end{tikzpicture}
  $}
  &\mapsto 2
\Gamma_{\bar b\bar b}^{(2)}
  \\
  \Gamma_{\bar 4i}^{(2)}:=
  \scalebox{.6}{$\omis\tripods  
      \begin{tikzpicture}
    \node[int] (i) at (0,.5) {};
    \node (v1) at (-.7,-.2) {$\omega$};
    \node (v2) at (-.2,-.2) {$\omega$};
    \node (v3) at (.2,-.2) {$\omega$};
    \node (v4) at (.7,-.2) {$i$};
  \draw (i) edge[crossed] (v1) edge (v2) edge (v3) edge (v4);
  \end{tikzpicture} $}
  &\mapsto 
  - \Gamma_{\bar bi}^{(2)} + (const) \Gamma_{i irr}^{(2)}
  \\
  \Gamma_{\bar B'}^{(2)}:=
  \scalebox{.6}{$\omis\tripods  
    \begin{tikzpicture}
    \node[int] (i) at (0,.5) {};
    \node[int] (j) at (1,.5) {};
    \node (v1) at (-.5,-.2) {$\omega$};
    \node (v2) at (0,-.2) {$\omega$};
    \node (v3) at (1,-.2) {$\omega$};
    \node (v4) at (1.5,-.2) {$\omega$};
    \node (v5) at (.5,-.2) {$\omega$};
  \draw (i) edge[crossed] (v1) edge (v2) edge (v5) edge (j) (j) edge (v3) edge (v4);
  \end{tikzpicture} $}
  &\mapsto
  \pm \Gamma_{\bar b\bar b}^{(2)}+(const) \Gamma_{birr}^{(2)}
  \\
  \Gamma_{b\bar 4}^{(2)}:=
  \scalebox{.6}{$\omis\tripods 
\begin{tikzpicture}
  \node[int] (i) at (0,.5) {};
  \node[int] (j) at (1,.5) {};
  \node (v1) at (-.5,-.2) {$\omega$};
  \node (v2) at (0,-.2) {$\omega$};
  \node (v3) at (1,-.2) {$\omega$};
  \node (v4) at (1.5,-.2) {$\omega$};
\draw (i) edge (v1) edge (v2) edge (j) (j) edge (v3) edge (v4);
\end{tikzpicture}
  \begin{tikzpicture}
    \node[int] (i) at (0,.5) {};
    \node (v1) at (-.7,-.2) {$\omega$};
    \node (v2) at (-.2,-.2) {$\omega$};
    \node (v3) at (.2,-.2) {$\omega$};
    \node (v4) at (.7,-.2) {$\omega$};
  \draw (i) edge[crossed] (v1) edge (v2) edge (v3) edge (v4);
  \end{tikzpicture}
$}
&\mapsto - \Gamma_{b\bar b}^{(2)} + (const) \Gamma_{birr}^{(2)}
\\
\Gamma_{\epsilon\bar b}^{(2)}
:=
\scalebox{.6}{$\omis\tripods 
  \begin{tikzpicture}
    \node (v) at (0,0) {$\omega$};
    \node (w) at (1,0) {$\epsilon$};
    \draw (v) edge (w);
  \end{tikzpicture}
  \begin{tikzpicture}
    \node[int] (i) at (0,.5) {};
    \node[int] (j) at (1,.5) {};
    \node (v1) at (-.5,-.2) {$\omega$};
    \node (v2) at (0,-.2) {$\omega$};
    \node (v3) at (1,-.2) {$\omega$};
    \node (v4) at (1.5,-.2) {$\omega$};
  \draw (i) edge (v1) edge (v2) edge[crossed] (j) (j) edge (v3) edge (v4);
  \end{tikzpicture}
$}
&\mapsto \sum_j \pm \Gamma_{j\bar b}^{(2)} + (const) \Gamma_{b\bar b}^{(2)}
\pm 4 \Gamma_{\bar b\bar b}^{(2)}
\end{align*}

\smallskip

\underline{Degree $23-n$:}
\[
\resizebox{.95\hsize}{!}{
$\begin{aligned}
\Gamma_{4\bar 4}^{(2)}:=\scalebox{.6}{$\omis\tripods 
\begin{tikzpicture}
  \node[int] (i) at (0,.5) {};
  \node (v1) at (-.6,-.2) {$\omega$};
  \node (v2) at (-.2,-.2) {$\omega$};
  \node (v3) at (.2,-.2) {$\omega$};
  \node (v4) at (.6,-.2) {$\omega$};
\draw (i) edge (v1) edge (v2) edge (v3) edge (v4);
\end{tikzpicture}
  \begin{tikzpicture}
    \node[int] (i) at (0,.5) {};
    \node (v1) at (-.7,-.2) {$\omega$};
    \node (v2) at (-.2,-.2) {$\omega$};
    \node (v3) at (.2,-.2) {$\omega$};
    \node (v4) at (.7,-.2) {$\omega$};
  \draw (i) edge[crossed] (v1) edge (v2) edge (v3) edge (v4);
  \end{tikzpicture}
$}
&\mapsto 3 \Gamma_{b\bar 4}^{(2)} + (const) \Gamma_{4b}^{(2)} + (const) \Gamma_{4irr}^{(2)}
\\
\Gamma^{(2)}_{\epsilon\bar 4}:=
\scalebox{.6}{$\omis\tripods 
  \begin{tikzpicture}
    \node (v) at (0,0) {$\omega$};
    \node (w) at (1,0) {$\epsilon$};
    \draw (v) edge (w);
  \end{tikzpicture}
  \begin{tikzpicture}
    \node[int] (i) at (0,.5) {};
    \node (v1) at (-.7,-.2) {$\omega$};
    \node (v2) at (-.2,-.2) {$\omega$};
    \node (v3) at (.2,-.2) {$\omega$};
    \node (v4) at (.7,-.2) {$\omega$};
  \draw (i) edge[crossed] (v1) edge (v2) edge (v3) edge (v4);
  \end{tikzpicture}
$}
&\mapsto - \Gamma_{\epsilon \bar b}^{(2)} + \sum_j \pm  \Gamma_{j\bar 4}^{(2)} 
+ (const) \Gamma_{b\bar 4}^{(2)} 
\pm \Gamma_{\bar B}^{(2)} \pm 3 \Gamma_{\bar B'}^{(2)}
\\
\Gamma_{\bar 5}^{(2)} := \scalebox{.6}{$\omis\tripods  
\begin{tikzpicture}
    \node[int] (i) at (0,.5) {};
    \node (v1) at (-.7,-.2) {$\omega$};
    \node (v2) at (-.2,-.2) {$\omega$};
    \node (v3) at (.2,-.2) {$\omega$};
    \node (v4) at (.7,-.2) {$\omega$};
    \node (v5) at (1.2,-.2) {$\omega$};
  \draw (i) edge[crossed] (v1) edge (v2) edge (v3) edge (v4) edge (v5);
  \end{tikzpicture} $}
  &\mapsto 6 \Gamma_{\bar B'}^{(2)} + (const) \Gamma_{4irr}^{(2)} + (const) \Gamma_{\bar B}^{(2)}
\end{aligned}$}
\]

We can now compute the cohomology of $\myB_{g,n}$ for $3g+2n=27$. The complex is concentrated in degrees $23-n$, $24-n$, $25-n$. Note that some generators exist only for higher genus or higher $n$. There is no cocycle of degree $23-n$, since the image of the differential is of full dimension. This is seen already by looking only at the leading terms $\Gamma_{b\bar 4}^{(2)}$, $\Gamma_{\epsilon \bar b}^{(2)}$, $\Gamma_{\bar B'}^{(2)}$.
Note that these generators only exist for $g$ large enough, i.e., $g\geq 4$ or $g\geq 7$ respectively. But if they do not (because $g$ is too small), the same holds for the corresponding generators of degree $23-n$.
Hence we conclude that $H^{23-n}(\myB_{g,n})=0$ for all $g,n$ considered.

Next we consider cocycles $x$ of degree $24-n$.
The general cocycle is a linear combination of the $9$ generators in degree $24-n$. By adding an exact term to $x$ we may however assume that this linear combination does not involve $\Gamma_{b\bar 4}^{(2)}$, $\Gamma_{\epsilon \bar b}^{(2)}$, $\Gamma_{\bar B'}^{(2)}$.
Assume first that $n\geq 1$. Then we claim that no linear combination of the remaining generators can be closed. This is so because the image of the generators under the differential is already of full rank if projected to the subspace spanned by the ``leading terms" $\Gamma_{birr}^{(2)}$, $\Gamma_{b\bar b}^{(2)}$,
$\Gamma_{j\bar b}^{(2)}$,
$\Gamma_{j,irr}^{(2)}$,
$\Gamma_{\bar b\bar b}^{(2)}$,
$\Gamma_{\bar bi}^{(2)}$.
Hence we have $H^{24-n}(\myB_{g,n})=0$ for $n\geq 1$.
In the special case $n=0$, i.e., $g=9$, the generator $\Gamma_{j,irr}^{(2)}$ of degree $25$ does not exist, while $\Gamma_{\epsilon,irr}^{(2)}$ does exist.
Hence a linear combination of $\Gamma_{\epsilon,irr}^{(2)}$ and $\Gamma_{4irr}^{(2)}$ is a cocycle, so that $H^{24}(\myB_{9,0})$ is one-dimensional.

Finally, we consider degree $25-n$. 
Any element $x$ of that degree is a cocyle. 
By adding exact terms we may ensure that $x$ does not involve the generators 
$\Gamma_{birr}^{(2)}$,
$\Gamma_{b\bar b}^{(2)}$,
$\Gamma_{j\bar b}^{(2)}$,
$\Gamma_{\bar b\bar b}^{(2)}$,
$\Gamma_{\bar bi}^{(2)}$,
and in addition we have to mod out the linear combination $\sum_j\pm \Gamma_{j,irr}^{(2)}$ (for $g>1$, $n>0$), that accounts for one copy of the sign representation $V_{1^n}$ of $\ss_n$.
Our $x$ can hence be a linear combination of the generators $\Gamma_i^{(2)}$,
$\Gamma_{j,irr}^{(2)}$ and $\Gamma_{ij}^{(2)}$.
The generators $\Gamma_i^{(2)}$ exist for $g\geq 3$ and $n\geq 1$ and contribute the representation $V_{1^n} \oplus V_{21^{n-2}}$ of $\bbS_n$.
As explained above the generators $\Gamma_{j,irr}^{(2)}$ and $\Gamma_{ij}^{(2)}$ in total contribute an $\ss_n$ representation $V_{21^{n-2}}$ if $g=1$, and a representation $V_{1^n}\oplus V_{21^{n-2}}$ for $g>1$, $n>0$. As mentioned above, in the case $g>1$ we have to remove the $V_{1^n}$ again (since it is in the image of the differential).
Hence we arrive at the following cohomology table:

\begin{align*}
  H^k(\myB_{1,12}) &=
  \begin{cases}
      V_{21^{10}} & \text{for $k=13$} \\
      0 & \text{otherwise}
  \end{cases}
  &
  H^k(\myB_{3,9}) &=
  \scalebox{.95}{$\begin{cases}
      V_{1^{9}}\oplus V_{21^7}\oplus V_{21^7} & \text{for $k=16$} \\
      0 & \text{otherwise}
  \end{cases}$
  }
  \\
  H^k(\myB_{5,6}) &=
  \begin{cases}
    V_{1^{6}}\oplus V_{21^4}\oplus V_{21^4} & \text{for $k=19$} \!\! \\
    0 & \text{otherwise}
\end{cases}
  &
  H^k(\myB_{7,3}) &=
  \begin{cases}
    V_{1^{3}}\oplus V_{21}\oplus V_{21} & \text{for $k=22$} \\
    0 & \text{otherwise}
\end{cases}
  \\
  H^k(\myB_{9,0}) &=
  \begin{cases}
    \mathbb C & \text{for $k=24$} \\
    0 & \text{otherwise.}
  \end{cases}
 \end{align*}

\noindent This completes the proof of Theorem~\ref{thm:lowexc13}.
