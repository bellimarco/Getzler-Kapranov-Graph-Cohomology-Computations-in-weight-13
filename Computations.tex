\section{Computations} \label{sec:Computations}

Our methodology is the following. Focusing on graphs of a specific excess, one edge group at a time, we compute the image of every graph under the differential and carry out gaussian elimination to determine simpler subspaces isomorphic to the kernel and the image.
After the simple subspaces have been factored out, we compute the cohomology.

Specifically, for every graph in an edge group we try to choose one non-zero term in its image such that the linear map defined by these choices has full rank. This means that the image of the differential is isomorphic through a change of basis to the subspace in the next edge group spanned by the chosen terms. Thus, to compute the cohomology, one can focus on the list of graphs with the chosen terms removed.

The choices of these leading terms have been hardcoded in the Python script as a dictionary.
The IDs of the pairs of graphs in this dictionary are displayed next to each other, with an arrow indicating which is the argument and which the image. Note that the ID is dependent upon the order in which the graphs have been generated and sorted throughout the script, so one has to be careful not to change those processes in order to use the same dictionary.

To avoid losing oneself in countless case distinctions, it is crucial that the choices of leading terms have matching $(g,n)$ range, so that for every $(g,n)$ pair they either both cancel out or both don't exist. It is also important to exploit the gaussian elimination process as much as possible by choosing to kill the graphs with the most complicated differential; these are the $A_3$ graphs and the ones with $\epsilon$ hairs.

\subsection{Previous results in excess 0,1,2} 
These cases have been computed in \cite[Section 3]{CLPW2}.

In excess smaller or equal to $0$, that is for $3g + 2n \leq 25$, we have $H^*(\myB_{g,n}) = 0$.

In excess 1 and 2 the cohomology is concentrated in the top degree of the complex, except in $(g,n)=(9,0)$ where it is in one degree lower.
\begin{align*}
    H^k(\myB_{2,10}) &=
    \begin{cases}
        V_{1^{10}} & \text{for $k=14$} \\
        0 & \text{otherwise}
    \end{cases}
&
H^k(\myB_{4,7}) &=
    \begin{cases}
        % V_{1^{7}}\oplus 
        V_{1^{7}} & \text{for $k=17$} \\
        0 & \text{otherwise}
    \end{cases}
\\
H^k(\myB_{6,4}) &=
\begin{cases}
    % V_{1^4} \oplus 
    V_{1^{4}} & \text{for $k=20$} \\
    0 & \text{otherwise}
\end{cases}     
&
H^k(\myB_{8,1}) &=
\begin{cases}
    % V_{1} \oplus 
    V_1 & \text{for $k=23$} \\
    0 & \text{otherwise.}
\end{cases}  
\end{align*}

\begin{align*}
  H^k(\myB_{1,12}) &=
  \begin{cases}
      V_{21^{10}} & \text{for $k=13$} \\
      0 & \text{otherwise}
  \end{cases}
  &
  H^k(\myB_{3,9}) &=
  \scalebox{.95}{$\begin{cases}
      V_{1^{9}}\oplus V_{21^7}^{\oplus 2} & \text{for $k=16$} \\
      0 & \text{otherwise}
  \end{cases}$
  }
  \\
  H^k(\myB_{5,6}) &=
  \begin{cases}
    V_{1^{6}}\oplus V_{21^4}^{\oplus 2} & \text{for $k=19$} \!\! \\
    0 & \text{otherwise}
\end{cases}
  &
  H^k(\myB_{7,3}) &=
  \begin{cases}
    V_{1^{3}}\oplus V_{21}^{\oplus 2} & \text{for $k=22$} \\
    0 & \text{otherwise}
\end{cases}
  \\
  H^k(\myB_{9,0}) &=
  \begin{cases}
    \mathbb C & \text{for $k=24$} \\
    0 & \text{otherwise.}
  \end{cases}
\end{align*}

Let us go through the excess two case again for the sake of continuity. There are 20 virtual blown-up graphs of excess 2 that we have to consider.

\textit{-- coming soon --}

%In $n=12$ only one of them exists, namely ID6 in our classification. Because of weight 13 relations it only contributes a $V_{21^{10}}$ factor.




\subsection{Excess 3} We are looking at the $(g,n)$ pairs $(2,11),(4,8),(6,5)$ and $(8,2)$.
There are 106 virtual blown-up graphs of excess 3 to consider, spanning four degree classes.

The two weight 13 relation groups with $12-n$ edges have basis (ID33, ID32) and (ID36, ID34) respectively.\\
The two weight 11 relation groups with $12-n$ edges have basis (ID60, ID62) and (ID67, ID69) respectively.\\
The two remaining weight 13 relation groups with $13-n$ edges have basis (ID38, ID35) and (ID39, ID37) respectively.\\
The above holds for every $(g,n)$ pair, when the relevant graphs exist. The choices of these bases have been made a posteriori to suit well the gaussian elimination process.

One checks that our choices of leading terms in the differential of each graph are well-posed and indipendent from one another. Only the eight following graphs remain after elimination, distributed in the last two edge groups $12-n$ and $13-n$; we draw the matrix of the differential between these two degrees.

\vspace{.5cm}
\begin{tabular}{ c c|c|c|c|c|c|c }
    & $12-n$ edges
    & \multicolumn{2}{c|}{ $\substack{ \text{ID32  n: 11, 8, 5, 2} \\
        \begin{tikzpicture}
            \node (w1) at (0,-.6) {\tiny$\omega$};
            \node (w2) at (.5,-.6) {\tiny$\epsilon$};
            \node[int] (v1) at (0.25,0) {};
            \node (j2) at (.25,.6) {\tiny$j$};
            \draw (v1) edge[crossed](w1) edge(w2)edge(j2);
        \end{tikzpicture}
        }$ }
    & \multicolumn{2}{c|}{ $\substack{ \text{ID67  n: 11, 8, 5, 2} \\
        \begin{tikzpicture}
            \node (w1) at (0,-.6) {\tiny$\epsilon$};
            \node (w2) at (.8,-.6) {\tiny$\omega$};
            \node[int] (v1) at (0.8,0) {};
            \node (j2) at (0,.4) {\tiny$j$};
            \draw (w1) edge(j2) (v1) edge(w2) edge[loop, crossed, distance=0.4cm](v1);
        \end{tikzpicture}
        }$ }
    & \multicolumn{2}{c}{ $\substack{ \text{ID122  n: 8, 5, 2} \\
        \begin{tikzpicture}
            \node (w1) at (0,-.5) {\tiny$\omega$};
            \node (w2) at (.4,-.5) {\tiny$\epsilon$};
            \draw (w1) edge[bend left=80, distance=.2cm] (w2);
            \node (w3) at (.8,-.6) {\tiny$\omega$};
            \node (w4) at (1.2,-.6) {\tiny$\omega$};
            \node (j2) at (1,.6) {\tiny$j$};
            \node[int] (v2) at (1,0) {};
            \draw(v2)edge(w3)edge(w4)edge(j2);
            \node (ww) at (1.6,-.6) {\tiny$\omega$};
            \node[int] (vv) at (1.6,0) {};
            \draw (vv) edge(ww) edge[loop, crossed, distance=0.4cm] (vv);
        \end{tikzpicture}
        }$ }
    % & \multicolumn{2}{c|}{ $\substack{ \text{ID26  n: 11, 8, 5, 2} \\
    %     \begin{tikzpicture}
    %         \node (w1) at (0,-.6) {\tiny$\omega$};
    %         \node (w2) at (.5,-.6) {\tiny$\omega$};
    %         \node[int] (v1) at (0.25,0) {};
    %         \node (j1) at (0,.6) {\tiny$i$};
    %         \node (j2) at (.5,.6) {\tiny$j$};
    %         \draw (v1) edge[crossed](w1)edge(j1) edge(w2)edge(j2);
    %     \end{tikzpicture}
    %     }$ }
    \\  $13-n$ edges  &
    & $V_{1^{n}}$ & $V_{21^{n-2}}$ & $V_{1^{n}}$ & $V_{21^{n-2}}$ & $V_{1^{n}}$ & $V_{21^{n-2}}$ 
    % & $V_{21^{n-2}}$ & $\substack{\text{\tiny$(n\geq 5)$\normalsize}\\ V_{31^{n-3}}}$
    \\[.1cm] \hline
    \multirow{5}{*}{$\substack{ \text{ID35  n: 11, 8, 5, 2 } \\
        \begin{tikzpicture}
            \node (w1) at (0,-.6) {\tiny$\omega$};
            \node (w2) at (.5,-.6) {\tiny$\omega$};
            \node[int] (v1) at (0,0) {};
            \node[int] (v2) at (.5,0) {};
            \node (j1) at (0,.6) {\tiny$i$};
            \node (j2) at (.5,.6) {\tiny$j$};
            \draw (v1) edge(v2) edge[crossed](w1)edge(j1) (v2)edge(w2)edge(j2);
        \end{tikzpicture}
        }$}
    & $V_{1^{n}}$ & \cellcolor{lightgray} & &  & & & \\ \cline{2-8}
    & $V_{21^{n-2}}$ & & \cellcolor{lightgray} & &  & & \\ \cline{2-8}
    & \tiny$(n\geq 5)$\normalsize $V_{221^{n-4}}$ & & & & & & \\ \cline{2-8}
    & \tiny$(n\geq 5)$\normalsize $V_{21^{n-2}}$ &  & & & & & \\ \cline{2-8}
    & \tiny$(n\geq 5)$\normalsize $V_{31^{n-3}}$ & & & & & & \\ \hline
    \multirow{2}{*}[-1mm]{$\substack{ \text{ID45  n: 11, 8, 5, 2 } \\
        \begin{tikzpicture}
            \node (w1) at (.25,-.6) {\tiny$\omega$};
            \node (w2) at (.75,-.6) {\tiny$\omega$};
            \node[int] (v1) at (.25,0) {};
            \node[int] (v2) at (.75,0) {};
            \node (j1) at (0,.6) {\tiny$i$};
            \node (j2) at (.5,.6) {\tiny$j$};
            \draw (v1) edge(w1)edge(j1)edge(j2) (v2)edge(w2) edge[loop, crossed, distance=0.4cm] (v2);
        \end{tikzpicture}
        }$}
    & $V_{21^{n-2}}$ & & & &  \cellcolor{lightgray} & & \\[.7cm] \cline{2-8}
    & \tiny$(n\geq 5)$\normalsize $V_{31^{n-3}}$ & &  & & & & \\[.7cm] \hline
    \multirow{2}{*}[-1mm]{$\substack{ \text{ID127  n: 8, 5, 2 } \\
        \begin{tikzpicture}
            \node (w1) at (0,-.6) {\tiny$\omega$};
            \node (w2) at (.4,-.6) {\tiny$\omega$};
            \node (j1) at (0.2,.6) {\tiny$i$};
            \node[int] (v1) at (0.2,0) {};
            \draw(v1)edge(w1)edge(w2)edge(j1);
            \node (w1) at (.8,-.6) {\tiny$\omega$};
            \node (w2) at (1.2,-.6) {\tiny$\omega$};
            \node (j2) at (1,.6) {\tiny$j$};
            \node[int] (v2) at (1,0) {};
            \draw(v2)edge(w3)edge(w4)edge(j2);
            \node (ww) at (1.5,-.6) {\tiny$\omega$};
            \node[int] (vv) at (1.5,0) {};
            \draw (vv) edge(ww) edge[loop, crossed, distance=0.4cm] (vv);
        \end{tikzpicture}
        }$}
    & $V_{21^{n-2}}$  & & & & & &  \cellcolor{lightgray} \\[.7cm] \cline{2-8}
    & \tiny$(n\geq 5)$\normalsize $V_{31^{n-3}}$  & & & & & & \\[.7cm]
    % \hline
    % \multirow{2}{*}[-1mm]{$\substack{ \text{ID45  n: 11, 8, 5, 2 } \\
    %     \begin{tikzpicture}
    %         \node (w1) at (0,-.6) {\tiny$\omega$};
    %         \node (w2) at (.5,-.6) {\tiny$\omega$};
    %         \node[int] (v1) at (0,0) {};
    %         \node[int] (v2) at (.5,0) {};
    %         \node (j1) at (0,.6) {\tiny$i$};
    %         \node (j2) at (.5,.6) {\tiny$j$};
    %         \draw (v1) edge[crossed](v2) edge(w1)edge(w2) (v2)edge(j1)edge(j2);
    %     \end{tikzpicture}
    %     }$}
    % & $V_{21^{n-2}}$ & \cellcolor{lightgray} & & & & & & & \\[.7cm] \cline{2-10}
    % & \tiny$(n\geq 5)$\normalsize $V_{31^{n-3}}$ & & \cellcolor{lightgray} & & & & & & \\[.7cm]
\end{tabular}
\vspace{.5cm}

A filled box in the above matrix means that the differential of the column has the row as a non-zero term. The graph in the second column actually has only one term in its differential, whereas the other two also have other terms, which have been removed through gaussian elimination.

We obtain the following cohomology groups. In particular, they are no longer concentrated in top degree, and as in excess 2, the pair of maximal genus $(8,2)$ vanishes in top degree.

\begin{align*}
    H^k(\myB_{2,11}) &=
    \scalebox{.95}{$\begin{cases}
        V_{1^{11}} & \text{for $k=14$} \\
        V_{21^{9}} \oplus V_{221^{7}} \oplus V_{31^{8}}^{\oplus 2} & \text{for $k=15$} \\
        0 & \text{otherwise}
    \end{cases}$}
    &
    H^k(\myB_{4,8}) &=
    \scalebox{.95}{$\begin{cases}
        V_{1^{8}}^{\oplus 2} & \text{for $k=17$} \\
        V_{21^{6}} \oplus V_{221^{4}} \oplus V_{31^{5}}^{\oplus 3} & \text{for $k=18$} \\
        0 & \text{otherwise}
    \end{cases}$}
    \\
    H^k(\myB_{6,5}) &=
    \scalebox{.95}{$\begin{cases}
        V_{1^{5}}^{\oplus 2} & \text{for $k=20$} \\
        V_{21^{3}} \oplus V_{221} \oplus V_{31^{2}}^{\oplus 3} & \text{for $k=21$} \\
        0 & \text{otherwise}
    \end{cases}$}
    &
    H^k(\myB_{8,2}) &=
    \scalebox{.95}{$\begin{cases}
      V_{1^{2}}^{\oplus 2} & \text{for $k=23$} \\
      0 & \text{otherwise}
    \end{cases}$}
\end{align*}


