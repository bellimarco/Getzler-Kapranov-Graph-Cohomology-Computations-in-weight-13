\section{Computer program}


\newcommand{\myB}{\bGK^{12,1}}

We use Python to generate all possible blown-up representations of the relevant decorated graphs with a certain upper excess bound $E_{max}$. The script is in the format of a Jupyter Notebook. The graphs are generated using Sage's builtin interface for the Nauty library and are later wrapped in Python objects. We use Pandas to store the large amounts of python objects in dataframes to take advantege of many useful data management tools like querying, sorting and grouping by the attributes of the objects. Finally, we use Matplotlib to either show the graphs in the output cells or save them in a pdf file. The project is saved in a GitHub repository \cite{github}.

\subsection{Generate blown-up components} \label{subsec:genBlownup} These are all the connected graphs with trivalent vertices, hairs labeled by $\epsilon,\omega$ or $j$ such that $0\leq 3(g-1)+3|\epsilon|+|\omega|+2|j| \leq E_{max}-25$, and falling under exactly one of the following cases:
\begin{enumerate}
    \item simple,
    \item simple and with a crossed $\omega$ hair,
    \item simple and with a crossed internal edge, possibly having maximum one multiple edge parallel to the crossed edge.
\end{enumerate}

We also add manually the following seven bonus graphs:
\[
    \begin{tikzpicture}
        \node (v) at (0,0) {$\omega$};
        \node (w) at (1,0) {$\omega$};
        \draw (v) edge[bend left=60, distance=.5cm] (w); 
    \end{tikzpicture}
    \hspace{0.4cm}
    \begin{tikzpicture}
        \node (v) at (0,0) {$\omega$};
        \node (w) at (1,0) {$\epsilon$};
        \draw (v) edge[bend left=60, distance=.5cm] (w); 
    \end{tikzpicture}
    \hspace{0.4cm}
    \begin{tikzpicture}
        \node (v) at (0,0) {$\epsilon$};
        \node (w) at (1,0) {$\epsilon$};
        \draw (v) edge[bend left=60, distance=.5cm] (w); 
    \end{tikzpicture}
    \hspace{0.4cm}
    \begin{tikzpicture}
        \node (w1) at (0,0) {$\omega$};
        \node (n) at (0,1.2) {$j$};
        \draw (w1) edge (n); 
    \end{tikzpicture}
    \hspace{0.4cm}
    \begin{tikzpicture}
        \node (w1) at (0,0) {$\epsilon$};
        \node (n) at (0,1.2) {$j$};
        \draw (w1) edge (n); 
    \end{tikzpicture}
    \hspace{0.4cm}
    \begin{tikzpicture}
        \node (w) at (0,0) {$\omega$};
        \node[int] (v) at (0,.8) {};
        \draw (w) edge (v);
        \draw (v) edge +(0,-.3) edge[loop, crossed, distance=0.8cm] (v);
    \end{tikzpicture}
    \hspace{0.4cm}
    \begin{tikzpicture}
        \node (w) at (0,0) {$\epsilon$};
        \node[int] (v) at (0,.8) {};
        \draw (w) edge (v) (v) edge[loop, crossed, distance=0.8cm] (v);
    \end{tikzpicture}
\]

In this list of blown-up graphs also lie many that will be later killed by relations, for example the lone hair with two $\omega$ labels, $A_2$ graphs with a loop at $\tilde{v}$ or a a second $\omega$ hair incident to the crossed hair. Nonetheless, it is useful to have all these graphs in the list for completeness, and verifying that the algorithms run correctly in full generality.

These graphs are wrapped in Python objects which compute at construction all graph parameters such as genus, valence at the special vertices, odd symmetries, Specht module contributions and plotting structures.

In order to make the task indipendent from the parameter $n$, we have to label the $n$ original hairs by the same token $j$. Every way of assigning the tokens $1,...,n$ to the $j$ hairs represents a generator, but not in a unique way due to the presence of automorphisms. Thus, each of the labeled graphs above actually contributes to the complex the subspace of all generators obtainable as an assignment on the $j$ hairs. We will work in terms of the irreducible representations of these subspaces, using the presentation of Specht modules defined by partitions of $n$.

\subsection{Weight 2 and $3b)$ relations} Case $B_1$ graphs are grouped by the isomorphism class of their contraction at the crossed edge.
This yields groups of blown-up components such that, any weight 2 cohomology relations or the $3b)$ relation restrict between graphs in the same group.

It is important to keep in mind that the $B_1$ graphs we generated in \ref{subsec:genBlownup} do not represent all decorated graphs with $\delta\{\substack{A\\A'}\}$ classes. We have ignored the ones with loops or multiple edges incident to just one end of the crossed edge. Because they are killed by the $3b$ relation or odd symmetries, they might also influence the relation group without appearing in our list.

There are three easy cases to handle automatically. If a weight 2 relation group has size one with an odd graph, then it vanishes. If a group has size one and the contraction at the crossed edge doesn't have loops nor multiple edges, then it vanishes if and only if the single graph in the group has an odd symmetry. If a group is of minimal valence $n_{\tilde{v}}=4$, then it vanishes if and only if any of its graphs have odd symmetries or its contraction at the crossed edge has loops or multiple edges. For higher group sizes or higher valence, we manually check each weight 2 relation group and hardcode a basis into the script.

\begin{remark} \label{rmk:weight2vanishing}
    A $B_1$ graph with a multiple edge and valence $4$ is equivalent to a $B_1$ graph with a loop, and thus vanishes because of relation $3b)$. More generally, we note that if there exist two hairs or half-edges $s,t\in N_{\tilde{v}}$ in a graph $G_{\bar{v},B}^{\tilde{v}_1,\tilde{v}_2}$ with the property that it vanishes whenever the partition $A\sqcup A'$ doesn't separate $s$ and $t$, then, for every other $x,y\in N_{\tilde{v}}$ distinct from $s,t$, the following relation holds:
\end{remark}
\[  
    \sum_{\substack{A\sqcup A'= N_{\tilde{v}}\\ s,x\in A, t,y\in A'}}
    \begin{tikzpicture}
        \node[int] (v) at (0,0) {};
        \node[int] (w) at (0.5,0) {};
        \node (s) at (-.1,.5)  {\tiny$s$};
        \node (t) at (.6,.5)  {\tiny$t$};
        \node (x) at (-.3,-.5) {\tiny$x$};
        \node (y) at (.8,-.5) {\tiny$y$};
        \draw (v) edge +(-.5,-.5) edge[dashed] +(-.5,.5) edge[dashed] +(-.5,.2) edge[dashed] +(-.5,-.2) edge[dashed] +(-.2,-.5) edge[crossed] (w) edge +(.1,.6)
        (w) edge +(.5,-.5) edge[dashed] +(.5,.5) edge[dashed] +(.5,.2) edge[dashed] +(.5,-.2) edge[dashed] +(.2,-.5) edge +(-.1,.6);
        \draw [decorate,decoration={brace,amplitude=5pt,mirror}]
        (-.6,.5) -- (-.6,-.5) node[midway,xshift=-1em]{$A$};
        \draw [decorate,decoration={brace,amplitude=5pt}]
        (1,.5) -- (1,-.5) node[midway,xshift=1em]{$A'$};
    \end{tikzpicture}
    =
    \sum_{\substack{A\sqcup A'= N_{\tilde{v}}\\ |A|,|A'|\geq 2 \\ x\in A, t,y\in A'}}
    \begin{tikzpicture}
        \node[int] (v) at (0,0) {};
        \node[int] (w) at (0.5,0) {};
        \node (t) at (.6,.5)  {\tiny$t$};
        \node (x) at (-.3,-.5) {\tiny$x$};
        \node (y) at (.8,-.5) {\tiny$y$};
        \draw (v) edge +(-.5,-.5) edge[dashed] +(-.5,.5) edge[dashed] +(-.5,.2) edge[dashed] +(-.5,-.2) edge[dashed] +(-.2,-.5) edge[crossed] (w) edge +(.1,.6)
        (w) edge +(.5,-.5) edge[dashed] +(.5,.5) edge[dashed] +(.5,.2) edge[dashed] +(.5,-.2) edge[dashed] +(.2,-.5) edge +(-.1,.6);
        \draw [decorate,decoration={brace,amplitude=5pt,mirror}]
        (-.6,.5) -- (-.6,-.5) node[midway,xshift=-1em]{$A$};
        \draw [decorate,decoration={brace,amplitude=5pt}]
        (1,.5) -- (1,-.5) node[midway,xshift=1em]{$A'$};
    \end{tikzpicture}
    =
    \sum_{\substack{A\sqcup A'= N_{\tilde{v}}\\ |A|,|A'|\geq 2 \\ x\in A, t,s\in A'}}
    \begin{tikzpicture}
        \node[int] (v) at (0,0) {};
        \node[int] (w) at (0.5,0) {};
        \node (s) at (.6,.5)  {\tiny$s$};
        \node (t) at (.4,.5)  {\tiny$t$};
        \node (x) at (-.3,-.5) {\tiny$x$};
        \draw (v) edge +(-.5,-.5) edge +(-.5,.5) edge[dashed] +(-.5,.2) edge[dashed] +(-.5,-.2) edge[dashed] +(-.2,-.5) edge[crossed] (w)
        (w) edge[dashed] +(.5,-.5) edge[dashed] +(.5,.5) edge[dashed] +(.5,.2) edge[dashed] +(.5,-.2) edge[dashed] +(.2,-.5) edge +(-.3,.6) edge +(.3,.6);
        \draw [decorate,decoration={brace,amplitude=5pt,mirror}]
        (-.6,.5) -- (-.6,-.5) node[midway,xshift=-1em]{$A$};
        \draw [decorate,decoration={brace,amplitude=5pt}]
        (1,.5) -- (1,-.5) node[midway,xshift=1em]{$A'$};
    \end{tikzpicture}
    =0  
\]
This observation can be employed to kill all $B_1$ graphs in excess 3 and 4 which have multiple edges and valence $n_{\tilde{v}}\geq 5$, except one. These are shown in \ref{equ:weight2killings}, and they come in three different weight 2 relation groups. They all vanish, except for the last ones, which merely have a relation between them.

\begin{multline} \label{equ:weight2killings}
    \begin{tikzpicture}
        \node[int] (v1) at (0,0) {};
        \node[int] (v2) at (1,0) {};
        \node (w1) at (-.25,-.9) {$\omega$};
        \node (w2) at (.25,-.9) {$\omega$};
        \node (w3) at (1,-.9) {$\omega$};
        \draw (v1) edge[crossed] (v2) edge[bend left=40, distance=0.6cm] (v2) edge (w1) edge (w2) (v2) edge (w3);
    \end{tikzpicture}
    =0  \hspace{2cm}
    \begin{tikzpicture}
        \node[int] (v1) at (0,0) {};
        \node[int] (v2) at (1,0) {};
        \node[int] (v3) at (1.4,0) {};
        \node (w1) at (0,-.9) {$\omega$};
        \node (w2) at (.8,-.9) {$\omega$};
        \node (w3) at (1.2,-.9) {$\omega$};
        \node (w4) at (1.8,-.9) {$\omega$};
        \draw (v1) edge[crossed] (v2) edge[bend left=40, distance=0.6cm] (v2) edge (w1);
        \draw (v2) edge (w2) edge (v3);
        \draw (v3) edge (w3) edge (w4);
    \end{tikzpicture}
    =0    \hspace{.8cm}
    \begin{tikzpicture}
        \node[int] (v1) at (0,0) {};
        \node[int] (v2) at (1,0) {};
        \node[int] (v3) at (1.4,0) {};
        \node (w1) at (0,-.9) {$\omega$};
        \node (w2) at (.4,-.9) {$\omega$};
        \node (w3) at (1.2,-.9) {$\omega$};
        \node (w4) at (1.8,-.9) {$\omega$};
        \draw (v1) edge[crossed] (v2) edge[bend left=40, distance=0.6cm] (v2) edge (w1) edge (w2);
        \draw (v2) edge (v3);
        \draw (v3) edge (w3) edge (w4);
    \end{tikzpicture}
    =0  \\
    \begin{tikzpicture}
        \node[int] (v1) at (0,0) {};
        \node[int] (v2) at (1,0) {};
        \node (w1) at (-.25,-.9) {$\omega$};
        \node (w2) at (.25,-.9) {$\omega$};
        \node (n) at (1.4,1) {$j$};
        \draw (v1) edge[crossed] (v2)  edge[bend left=40, distance=0.6cm] (v2) edge (w1) edge (w2);
        \draw (v2) edge[bend right=30](n);
    \end{tikzpicture}
    =0 \hspace{.8cm}
    \begin{tikzpicture}
        \node[int] (v1) at (0,0) {};
        \node[int] (v2) at (1,0) {};
        \node (w1) at (0,-.9) {$\omega$};
        \node (w2) at (1,-.9) {$\omega$};
        \node (n) at (1.4,1) {$j$};
        \draw (v1) edge[crossed] (v2) edge[bend left=40, distance=0.6cm] (v2) edge (w1);
        \draw (v2) edge[bend right=30](n) edge (w2);
    \end{tikzpicture}
    =0 \hspace{2cm}
    \begin{tikzpicture}
        \node[int] (v1) at (0,0) {};
        \node[int] (v2) at (1,0) {};
        \node (w1) at (-.4,-.9) {$\omega$};
        \node (w2) at (.4,-.9) {$\omega$};
        \node (w3) at (1,-.9) {$\omega$};
        \node (w4) at (0,-.9) {$\omega$};
        \draw (v1) edge[crossed] (v2) edge[bend left=40, distance=0.6cm] (v2) edge (w1) edge (w2) edge (w4);
        \draw (v2) edge (w3);
    \end{tikzpicture}
    +
    \begin{tikzpicture}
        \node[int] (v1) at (0,0) {};
        \node[int] (v2) at (1,0) {};
        \node (w1) at (-.25,-.9) {$\omega$};
        \node (w2) at (.25,-.9) {$\omega$};
        \node (w3) at (.75,-.9) {$\omega$};
        \node (w4) at (1.25,-.9) {$\omega$};
        \draw (v1) edge[crossed] (v2) edge[bend left=40, distance=0.6cm] (v2) edge (w1) edge (w2);
        \draw (v2) edge (w3) edge (w4);
    \end{tikzpicture}
    =0
\end{multline}

Remark \ref{rmk:weight2vanishing} can also be used to kill other graphs, such as ones where the crossed edge is part of a triangle or where two half-edges being in at the same endpoint would create an odd symmetry:

\begin{equation}
    \begin{tikzpicture}
        \node[int] (v1) at (0,0) {};
        \node[int] (v2) at (1,0) {};
        \node[int] (v3) at (.5,.5) {};
        \node (w1) at (-.25,-.9) {$\omega$};
        \node (w2) at (.25,-.9) {$\omega$};
        \node (w3) at (.9,-.7) {$\omega$};
        \node (w4) at (1.25,-.9) {$\omega$};
        \draw (v1) edge[crossed] (v2) edge (v3) edge (w1) edge (w2);
        \draw (v2) edge (v3) edge (w4);
        \draw (v3) edge (w3);
    \end{tikzpicture}
    =0  \hspace{1cm}
    \begin{tikzpicture}
        \node[int] (v1) at (0,0) {};
        \node[int] (v2) at (1,0) {};
        \node[int] (v3) at (1.75,0) {};
        \node[int] (v4) at (2.5,0) {};
        \node (w1) at (-.25,-.9) {$\omega$};
        \node (w2) at (.25,-.9) {$\omega$};
        \node (w3) at (.75,-.9) {$\omega$};
        \node (w4) at (1.25,-.9) {$\omega$};
        \node (w5) at (1.75,-.9) {$\omega$};
        \node (w6) at (2.25,-.9) {$\omega$};
        \node (w7) at (2.75,-.9) {$\omega$};
        \draw (v1) edge(v2)  edge (w1) edge (w2) (v2) edge[crossed] (v3)  edge (w3) edge (w4);
        \draw (v3) edge (v4)  edge (w5) (v4) edge(w6)edge(w7);
    \end{tikzpicture}
    =0.
\end{equation}

\ref{equ:weight2groupof4} is another example in excess 4 of a weight 2 relation group. It contains the four graphs which appear as terms in the equation, and one checks that all the relations turn out equivalent to the one displayed. Thus, a basis for this group is determined by any three graphs.
\begin{equation} \label{equ:weight2groupof4}
    \begin{tikzpicture}
        \node[int] (v1) at (0,0) {};
        \node[int] (v2) at (.75,0) {};
        \node[int] (v3) at (1.5,0) {};
        \node (j) at (.75,.9) {\small$j$};
        \node (w1) at (-.4,-.8) {\small$\omega$};
        \node (w2) at (0,-.8) {\small$\omega$};
        \node (w3) at (.4,-.8) {\small$\omega$};
        \node (w4) at (1.3,-.8) {\small$\omega$};
        \node (w5) at (1.7,-.8) {\small$\omega$};
        \draw (v1) edge[crossed](v2) edge(w1)edge(w2)edge(w3);
        \draw (v2) edge(v3) edge(j)  (v3) edge(w4)edge(w5);
    \end{tikzpicture} +
    \begin{tikzpicture}
        \node[int] (v1) at (0,0) {};
        \node[int] (v2) at (.75,0) {};
        \node[int] (v3) at (1.5,0) {};
        \node (j) at (.75,.9) {\small$j$};
        \node (w1) at (-.2,-.8) {\small$\omega$};
        \node (w2) at (.2,-.8) {\small$\omega$};
        \node (w3) at (.75,-.8) {\small$\omega$};
        \node (w4) at (1.3,-.8) {\small$\omega$};
        \node (w5) at (1.7,-.8) {\small$\omega$};
        \draw (v1) edge[crossed](v2) edge(w1)edge(w2);
        \draw (v2) edge(v3) edge(w3) edge(j)  (v3) edge(w4)edge(w5);
    \end{tikzpicture} -
    \begin{tikzpicture}
        \node[int] (v1) at (0,0) {};
        \node[int] (v2) at (.75,0) {};
        \node[int] (v3) at (1.5,0) {};
        \node (j) at (1.5,.9) {\small$j$};
        \node (w1) at (-.2,-.8) {\small$\omega$};
        \node (w2) at (.2,-.8) {\small$\omega$};
        \node (w3) at (.75,-.8) {\small$\omega$};
        \node (w4) at (1.3,-.8) {\small$\omega$};
        \node (w5) at (1.7,-.8) {\small$\omega$};
        \draw (v1) edge(v2) edge(w1)edge(w2);
        \draw (v2) edge[crossed](v3) edge(w3)  (v3) edge(w4)edge(w5) edge(j);
    \end{tikzpicture} -
    \begin{tikzpicture}
        \node[int] (v1) at (0,0) {};
        \node[int] (v2) at (.75,0) {};
        \node[int] (v3) at (1.5,0) {};
        \node (j) at (1.5,.9) {\small$j$};
        \node (w1) at (-.4,-.8) {\small$\omega$};
        \node (w2) at (.1,-.8) {\small$\omega$};
        \node (w3) at (.5,-.8) {\small$\omega$};
        \node (w4) at (1,-.8) {\small$\omega$};
        \node (w5) at (1.5,-.8) {\small$\omega$};
        \draw (v1) edge(v2) edge(w1)edge(w2);
        \draw (v2) edge[crossed](v3) edge(w3)edge(w4)  (v3) edge(w5) edge(j);
    \end{tikzpicture}
    =0
\end{equation}


\subsection{Generating virtual blown-up graphs}  First we remove odd symmetries, weight 2 and $3b)$ relations, and $A_2$ graphs which are related to $B_{irr}$ graphs through relation $4$. Then, we build so called \textit{virtual blown-up graphs}. These are all unordered lists of the remaining blown-up components of excess greater or equal to one which have exactly one crossed component, total number of $\omega$ hairs less than or equal to $11$, excesses $e_i$ summing up to less than or equal to $E_{max}-25$, and which don't contain more than once components that create an odd symmetry when exchanged by an automorphism. We explain this last condition.

Two isomorphic blown-up components (which don't individually have an odd symmetry already) create an odd symmetry if and only if they have no $j$ hairs (because automorphisms don't exchange the $j$ hairs) and have an odd number of internal edges plus $\epsilon$ hairs. Also the lone hair with two $\epsilon$ labels is of this type.

Virtual blown-up graphs are likewise wrapped in a Python object and stored in a dataframe, along with aggregate quantities from the list of blown-up components. Note that these lists have an excess value but don't yet determine a unique generator, as we haven't pinned down which $(g,n)$ pair of the excess class we are considering, hence the name \textit{virtual}. The possible $(g,n)$ pairs are determined by how many lone $j$ hairs vs tripods it is possible to append to the list of components, they are called the \textit{(g,n)-range} of the virtual blown-up graph.



\subsection{Weight 11 relations} The first step of the program is actually to generate so called \textit{unmarked blown-up components}, which have a unique label $u$ for the half-edges that where incident to $\bar{v}$. Only afterwards, blown-up components with $\omega$ and $\epsilon$ hairs are generated by marking the $u$ hairs in every possible way. The unmarked component which was used as a template for the marking is stored in each respective blown-up component. In so doing, a list of blown-up components can be obtained from another by permuting its $\omega$ and $\epsilon$ hairs if and only if they have the same list of unmarked components and same number of total $\epsilon$ hairs, which can be instantly checked instead of testing for isomorphism.

Thus, we group by the list of unmarked components and number of $\epsilon$ hairs to determine weight 11 relations groups of $B$ and $B_{irr}$ blown-up graphs.
In addition, this can be done at the virtual level without pinning down a $(g,n)$ pair in the excess class by ignoring components obtained as a marking of a lone $j$ hair or a tripod.

For example, the unmarked component in \ref{equ:weight11relationexample} determines in excess 3 (the excess determines the number of $\epsilon$ hairs) a weight 11 relation group of three virtual blown-up graphs. Note that not all graphs in the same relation group must have the same $(g,n)$-range: the first and third exist for $(4,8),(6,5),(8,2)$, whereas the middle one only for $(6,5),(8,2)$.
The corresponding weight 11 relation is an integer-weighted alternating sum of the three or two graphs, depending upon the $(g,n)$ pair. Thus, discarding the third graph leaves us with a basis of the relation group for all three $(g,n)$ pairs.
\begin{equation} \label{equ:weight11relationexample}
    \begin{tikzpicture}
        \node (w1) at (0,-.5) {$u$};
        \node (w2) at (.5,-.5) {$u$};
        \node (w3) at (1,-.5) {$u$};
        \node (w4) at (1.5,-.5) {$u$};
        \node[int] (v1) at (.25,.5) {};
        \node[int] (v2) at (1.25,.5) {};
        \draw(v1) edge[crossed](v2)edge(w1)edge(w2) (v2) edge(w3)edge(w4);
    \end{tikzpicture}
    \xlongrightarrow{\substack{\text{marked components}\\\text{with one $\epsilon$ hair}}} \hspace{.8cm}
    \begin{tikzpicture}
        \node (w1) at (0,-.5) {$\omega$};
        \node (w2) at (.5,-.5) {$\omega$};
        \node (w3) at (1,-.5) {$\omega$};
        \node (w4) at (1.5,-.5) {$\omega$};
        \node[int] (v1) at (.25,.5) {};
        \node[int] (v2) at (1.25,.5) {};
        \draw(v1) edge[crossed](v2)edge(w1)edge(w2) (v2) edge(w3)edge(w4);
        \node (e) at (-.6,-.5) {$\epsilon$};
        \node (j) at (-.6,.7) {$j$};
        \draw(e) edge(j);
    \end{tikzpicture},
    \begin{tikzpicture}
        \node (w1) at (0,-.5) {$\omega$};
        \node (w2) at (.5,-.5) {$\omega$};
        \node (w3) at (1,-.5) {$\omega$};
        \node (w4) at (1.5,-.5) {$\omega$};
        \node[int] (v1) at (.25,.5) {};
        \node[int] (v2) at (1.25,.5) {};
        \draw(v1) edge[crossed](v2)edge(w1)edge(w2) (v2) edge(w3)edge(w4);
        \node (ww1) at (-1.5,-.5) {$\omega$};
        \node (ww2) at (-1,-.5) {$\omega$};
        \node (ww3) at (-.5,-.5) {$\epsilon$};
        \node[int] (vv) at (-1,.5) {};
        \draw(vv) edge(ww1)edge(ww2)edge(ww3);
    \end{tikzpicture},
    \begin{tikzpicture}
        \node (w1) at (0,-.5) {$\omega$};
        \node (w2) at (.5,-.5) {$\omega$};
        \node (w3) at (1,-.5) {$\omega$};
        \node (w4) at (1.5,-.5) {$\epsilon$};
        \node[int] (v1) at (.25,.5) {};
        \node[int] (v2) at (1.25,.5) {};
        \draw(v1) edge[crossed](v2)edge(w1)edge(w2) (v2) edge(w3)edge(w4);
    \end{tikzpicture}
\end{equation}

\subsection{Weight 13 relations} The process is similar to weight 11, only that now we have to determine virtual blown-up graphs up to permutation of the $\omega$ hairs with the two hairs or half-edges at $\tilde{v}$. This is done again by blowing up at $\tilde{v}$:  the crossed edge and $\tilde{v}$ are discarded, the two newly created hairs are marked $\omega$, and we end up with a list of uncrossed blown-up components with a total of 12 $\omega$ hairs.

Thus, we group by the list of so called \textit{$A_2$ components} to determine weight 13 relation groups of $A_2$ blown-up graphs. Again, this can be done at the virtual level by ignoring the lone $\omega-j$ hairs and the tripods with $\omega$ labels.

For example, the two blown-up components in \ref{equ:weight13relationexample} determine in excess 4 a weight 13 relation group of seven virtual blown-up graphs. One can check that they all have the same list of blown-up components after removing $\tilde{v}$ (up to lone hairs and tripods).  The graphs with two tripods glued to $\tilde{v}$ (or with two hairs of the same tripod) are not listed because they would have an odd symmetry. Note again that not all blown-up graphs have the same $(g,n)$-range.

\begin{multline} \label{equ:weight13relationexample}
    \begin{tikzpicture}
        \node (w1) at (0,-.5) {\small$\omega$};
        \node (w2) at (.4,-.5) {\small$\omega$};
        \node (w3) at (.8,-.5) {\small$\omega$};
        \node (w4) at (1.2,-.5) {\small$\omega$};
        \node[int] (v1) at (.2,.5) {};
        \node[int] (v2) at (1,.5) {};
        \draw(v1) edge(v2)edge(w1)edge(w2) (v2) edge(w3)edge(w4);
        \node (w5) at (1.6,-.5) {\small$\omega$};
        \node (w6) at (2,-.5) {\small$\omega$};
        \node[int] (v5) at (1.8,.3) {};
        \node (j) at (1.8,1.2) {\small $j$};
        \draw (v5) edge(j)edge(w5)edge(w6);
    \end{tikzpicture}
    \xlongrightarrow{\text{$A_2$ blown-ups}}
    \begin{tikzpicture}
        \node (w1) at (0,-.5) {\small$\omega$};
        \node (w2) at (.4,-.5) {\small$\omega$};
        \node (w3) at (.8,-.5) {\small$\omega$};
        \node[int] (v1) at (.2,.5) {};
        \node[int] (v2) at (1,.5) {};
        \draw(v1) edge(v2)edge(w1)edge(w2) (v2) edge(w3);
        \node (w6) at (2.4,-.5) {\small$\omega$};
        \node[int] (v5) at (2.2,.3) {};
        \node (j) at (2.2,1.2) {\small $j$};
        \draw (v5) edge(j)edge(w6);
        \node (ww) at (1.6,-.5) {\small$\omega$};
        \node[int] (CR) at (1.6,.3) {};
        \draw (CR) edge[crossed](ww) edge(v2)edge(v5);
    \end{tikzpicture},
    \begin{tikzpicture}
        \node (w1) at (0,-.5) {\small$\omega$};
        \node (w2) at (.4,-.5) {\small$\omega$};
        \node (w3) at (.8,-.5) {\small$\omega$};
        \node[int] (v1) at (.2,.5) {};
        \node[int] (v2) at (1,.5) {};
        \draw(v1) edge(v2)edge(w1)edge(w2) (v2) edge(w3);
        \node (w5) at (2,-.5) {\small$\omega$};
        \node (w6) at (2.4,-.5) {\small$\omega$};
        \node[int] (v5) at (2.2,.3) {};
        \node (j) at (2.2,1.2) {\small $j$};
        \draw (v5) edge(j)edge(w5)edge(w6);
        \node (ww) at (1.6,-.5) {\small$\omega$};
        \node[int] (CR) at (1.6,.3) {};
        \node (j1) at (1.6,1.2) {\small $j$};
        \draw (CR) edge[crossed](ww) edge(v2)edge(j1);
    \end{tikzpicture},
    \begin{tikzpicture}
        \node (w1) at (0,-.5) {\small$\omega$};
        \node (w2) at (.4,-.5) {\small$\omega$};
        \node (w3) at (.8,-.5) {\small$\omega$};
        \node[int] (v1) at (.2,.5) {};
        \node[int] (v2) at (1,.5) {};
        \draw(v1) edge(v2)edge(w1)edge(w2) (v2) edge(w3);
        \node (w5) at (2.8,-.5) {\small$\omega$};
        \node (w6) at (3.2,-.5) {\small$\omega$};
        \node[int] (v5) at (3,.3) {};
        \node (j) at (3,1.2) {\small $j$};
        \draw (v5) edge(j)edge(w5)edge(w6);
        \node (ww) at (1.6,-.5) {\small$\omega$};
        \node[int] (CR) at (1.6,.3) {};
        \node (ww1) at (2,-.5) {$\omega$};
        \node (ww2) at (2.4,-.5) {$\omega$};
        \node[int] (vv) at (2,.3) {};
        \draw(vv) edge(ww1)edge(ww2);
        \draw (CR) edge[crossed](ww) edge(v2)edge(vv);
    \end{tikzpicture}, \\
    \begin{tikzpicture}
        \node (w1) at (0,-.5) {\small$\omega$};
        \node (w2) at (.4,-.5) {\small$\omega$};
        \node (w3) at (.8,-.5) {\small$\omega$};
        \node (w4) at (1.2,-.5) {\small$\omega$};
        \node[int] (v1) at (.2,.5) {};
        \node[int] (v2) at (1,.5) {};
        \draw(v1) edge(v2)edge(w1)edge(w2) (v2) edge(w3)edge(w4);
        \node (w6) at (2.4,-.5) {\small$\omega$};
        \node[int] (v5) at (2.2,.3) {};
        \node (j) at (2.2,1.2) {\small $j$};
        \draw (v5) edge(j)edge(w6);
        \node (ww) at (1.6,-.5) {\small$\omega$};
        \node[int] (CR) at (1.6,.3) {};
        \node (j1) at (1.6,1.2) {\small $j$};
        \draw (CR) edge[crossed](ww) edge(j1)edge(v5);
    \end{tikzpicture},
    \begin{tikzpicture}
        \node (w1) at (0,-.5) {\small$\omega$};
        \node (w2) at (.4,-.5) {\small$\omega$};
        \node (w3) at (.8,-.5) {\small$\omega$};
        \node (w4) at (1.2,-.5) {\small$\omega$};
        \node[int] (v1) at (.2,.5) {};
        \node[int] (v2) at (1,.5) {};
        \draw(v1) edge(v2)edge(w1)edge(w2) (v2) edge(w3)edge(w4);
        \node (w6) at (3.2,-.5) {\small$\omega$};
        \node[int] (v5) at (3,.3) {};
        \node (j) at (3.2,1.2) {\small $j$};
        \draw (v5) edge(j)edge(w6);
        \node (ww1) at (1.6,-.5) {$\omega$};
        \node (ww2) at (2,-.5) {$\omega$};
        \node[int] (vv) at (2,.3) {};
        \draw(vv) edge(ww1)edge(ww2);
        \node (ww) at (2.4,-.5) {\small$\omega$};
        \node[int] (CR) at (2.4,.3) {};
        \draw (CR) edge[crossed](ww) edge(vv)edge(v5);
    \end{tikzpicture},
    \begin{tikzpicture}
        \node (w1) at (0,-.5) {\small$\omega$};
        \node (w2) at (.4,-.5) {\small$\omega$};
        \node (w3) at (.8,-.5) {\small$\omega$};
        \node (w4) at (1.2,-.5) {\small$\omega$};
        \node[int] (v1) at (.2,.5) {};
        \node[int] (v2) at (1,.5) {};
        \draw(v1) edge(v2)edge(w1)edge(w2) (v2) edge(w3)edge(w4);
        \node (w5) at (2.8,-.5) {\small$\omega$};
        \node (w6) at (3.2,-.5) {\small$\omega$};
        \node[int] (v5) at (3,.3) {};
        \node (j) at (3,1.2) {\small $j$};
        \draw (v5) edge(j)edge(w5)edge(w6);
        \node (ww1) at (1.6,-.5) {$\omega$};
        \node (ww2) at (2,-.5) {$\omega$};
        \node[int] (vv) at (2,.3) {};
        \draw(vv) edge(ww1)edge(ww2);
        \node (jj) at (2.4,1.2) {\small$j$};
        \node (ww) at (2.4,-.5) {\small$\omega$};
        \node[int] (CR) at (2.4,.3) {};
        \draw (CR) edge[crossed](ww) edge(vv) edge(jj);
    \end{tikzpicture},
    \begin{tikzpicture}
        \node (w1) at (0,-.5) {\small$\omega$};
        \node (w2) at (.4,-.5) {\small$\omega$};
        \node (w3) at (.8,-.5) {\small$\omega$};
        \node (w4) at (1.2,-.5) {\small$\omega$};
        \node[int] (v1) at (.2,.5) {};
        \node[int] (v2) at (1,.5) {};
        \draw(v1) edge(v2)edge(w1)edge(w2) (v2) edge(w3)edge(w4);
        \node (w5) at (2,-.5) {\small$\omega$};
        \node (w6) at (2.4,-.5) {\small$\omega$};
        \node[int] (v5) at (2.2,.3) {};
        \node (j) at (2.2,1.2) {\small $j$};
        \draw (v5) edge(j)edge(w5)edge(w6);
        \node (jj) at (1.6,1.2) {\small$j$};
        \node (jjj) at (1.2,1.2) {\small$j$};
        \node (ww) at (1.6,-.5) {\small$\omega$};
        \node[int] (CR) at (1.6,.3) {};
        \draw (CR) edge[crossed](ww) edge(jjj) edge(jj);
    \end{tikzpicture},
\end{multline}



\subsection{Completing virtual blown-up graphs} The script also offers the possibility to focus on a particular $(g,n)$ pair by \textit{completing} each blown-up graph in an excess class by adding the right amount of $\omega-j$ hairs and tripods to the list of blown-up components.

Having at disposal the Specht module contributions of each generator of the complex, we can compute the Euler Characteristic for each $(g,n)$ pair individually, but with a caveat: we haven't resolved the weight 11 and 13 relations. We have checked that, after removing the redundant contributions, the Euler Characteristic in excesses smaller or equal to 3 matches the one computed in \cite[Figure 1]{CLPW}.

\subsection{Reading the data} $\omega,\epsilon$ and $j$ hairs are colored blue, yellow and green respectively. The crossed edge is colored red.

The virtual or completed blown-up graphs can be printed by excess or by the $(g,n)$ pair respectively. They are further grouped by the amount of internal edges (which for virtual blown-ups can be written dependently on $n$) and the cases $A_3,A_2,B_1,B_{irr}$.

Over each blown-up graph is written a unique identifier, the Specht module contribution of the graph and, for virtual blown-ups, the values of $n$ for which the graph can be completed to a $(g,n)$ graph of the given excess. Note that the actual Specht module contribution might not be accurate for graphs in non-trivial relation groups; it has to be computed manually for each group as a whole. 

In addition, over most graphs we write the ID of the chosen term for the gaussian elimination process; this is explained in \ref{sec:Computations}.

The start and endpoint of the list of components of each blown-up graph is distinguished by the crossed component, which is always displayed rightmost within its list. In the cases $A_2, B_1$ and $B_{irr}$, graphs in the same relation group are written side by side and on multiple consecutive rows if necessary. When a relation group ends, the next one starts on the row below. Since there are many weight 11 relation groups of size one, which thus form an indipendent set, we display them all in the same rows without line breaking, as we do for graphs in case $A_3$.