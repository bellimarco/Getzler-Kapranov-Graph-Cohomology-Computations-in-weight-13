



\section{Introduction}

The authors of \cite{CLPW2} study the weight graded pieces of $H^*_c(\M_{g,n})$ using graph complexes. The associated graded of the weight filtration is identified with the modular cooperad whose $(g,n)$ part is $H^*(\Mb_{g,n})$, it is known as the Getzler-Kapranov graph complex:
\begin{equation}
    \GK_{g,n} \cong  \gr^W_k H^*_c(\M_{g,n}) := W_k H^*_c(\M_{g,n})/W_{k-1}H^*_c(\M_{g,n})
\end{equation}


\begin{prop}\label{prop:wt 13 vanishing}
    If $3g + 2n \leq 25$ then $\gr_{13}^W H^*_c(\M_{g,n}) = 0$.
\end{prop}

\begin{thm} \label{thm:lowexc13}
    Suppose $3g + 2n \in \{26, 27\}$. Then $\gr_{13}^W H^*_c(\M_{g,n})$ is nonzero only in degree $$k(g,n) = 3g + n - 2 - \delta_{0,n},$$ and there is an $\ss_n$-equivariant isomorphism $\gr_{13}^W H^{k(g,n)}_c(\cM_{g,n}) \cong Z_{g,n} \otimes \lstw$, where
\begin{align*}
    Z_{1,12} & \cong V_{21^{10}} & Z_{2,10} & \cong V_{1^{10}} & Z_{3,9} & \cong V_{1^{9}} \\
    Z_{4,7} & \cong V_{1^{7}} & Z_{5,6} & \cong V_{1^{6}} \oplus V_{21^4}^{\oplus 2} & Z_{6,4} & \cong V_{1^{4}} \\ Z_{7,3} & \cong V_{1^{3}} \oplus V_{21}^{\oplus 2}& Z_{8,1} & \cong \Q & Z_{9,0} & \cong \Q 
\end{align*}
In particular, we have $\gr_{13}^W H^{24}_c(\M_9) \cong \lstw$.
\end{thm}

\noindent  Note that  $\gr_2^W H^{24}_c(\M_9)$ is also nonzero; the shift of $W_0H^6_c(\M_3) \wedge W_0H^{15}_c(\M_6)$ appearing in the first term of  \cite[Theorem~1.2]{PayneWillwacher21} contributes a summand isomorphic to $\l$. 