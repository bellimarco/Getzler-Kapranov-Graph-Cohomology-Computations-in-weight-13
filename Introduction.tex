



\section{Introduction}

The authors of \cite{CLPW2} study the weight graded pieces of $H^*_c(\M_{g,n})$ using graph complexes. The associated graded of the weight filtration is identified with the cohomology of the modular cooperad whose $(g,n)$ part is $H^*(\Mb_{g,n})$, it is known as the Getzler-Kapranov graph complex:
\begin{equation}
    H^*(\GK_{g,n}^k) \cong  \gr^W_k H^*_c(\M_{g,n}) := W_k H^*_c(\M_{g,n})/W_{k-1}H^*_c(\M_{g,n})
\end{equation}

They obtain two following results in weight 13.
\begin{prop}\label{prop:wt 13 vanishing}
    If $3g + 2n \leq 25$ then $\gr_{13}^W H^*_c(\M_{g,n}) = 0$.
\end{prop}

\begin{thm} \label{thm:lowexc13}
    Suppose $3g + 2n \in \{26, 27\}$. Then $\gr_{13}^W H^*_c(\M_{g,n})$ is nonzero only in degree $$k(g,n) = 3g + n - 2 - \delta_{0,n},$$ and there is an $\ss_n$-equivariant isomorphism $\gr_{13}^W H^{k(g,n)}_c(\cM_{g,n}) \cong Z_{g,n} \otimes \lstw$, where
\begin{align*}
    Z_{1,12} & \cong V_{21^{10}} & Z_{2,10} & \cong V_{1^{10}} & Z_{3,9} & \cong V_{1^{9}} \\
    Z_{4,7} & \cong V_{1^{7}} & Z_{5,6} & \cong V_{1^{6}} \oplus V_{21^4}^{\oplus 2} & Z_{6,4} & \cong V_{1^{4}} \\ Z_{7,3} & \cong V_{1^{3}} \oplus V_{21}^{\oplus 2}& Z_{8,1} & \cong \Q & Z_{9,0} & \cong \Q 
\end{align*}
\end{thm}

With our computations we extend to the following case.
\begin{thm} \label{thm:excess28}
    Suppose $3g + 2n =28$. Then $\gr_{13}^W H^*_c(\M_{g,n})$ vanishes outside the degrees $$k_1(g,n) = 3g + n - 2 \hspace{.5cm}\text{and}\hspace{.5cm} k_2(g,n) = 3g + n - 3$$ and there are $\ss_n$-equivariant isomorphisms $$ \gr_{13}^W H^{k_1(g,n)}_c(\cM_{g,n}) \cong Z_{g,n} \otimes \lstw  \hspace{1cm} \gr_{13}^W H^{k_2(g,n)}_c(\cM_{g,n}) \cong W_{g,n} \otimes \lstw , $$ where
\begin{align*}
    Z_{2,11} &\cong V_{21^{9}}\oplus V_{221^7} \oplus V_{31^8}^{\oplus 2} & Z_{4,8} &\cong V_{21^6}\oplus V_{221^4}\oplus V_{31^5}^{\oplus 3} & Z_{6,5} &\cong V_{21^3}\oplus V_{221}\oplus V_{31^2}^{\oplus 3}
\end{align*}
\begin{align*}
    Z_{8,2}&=0 &  W_{2,11}&=V_{1^{11}}^{\oplus 2} & W_{4,8}&=V_{1^8}^{\oplus 2} & W_{6,5}&=V_{1^5}^{\oplus 2} & W_{8,2}&=V_{1^2}
\end{align*}
\end{thm}