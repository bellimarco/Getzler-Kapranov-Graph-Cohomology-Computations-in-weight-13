\newpage
\section{Appendix} \label{sec:appendix}

The $\Q$-Hodge structure on the cohomology of the moduli space of curves delivers the following decompositions.
\begin{equation}\label{thm:GroupsHodgeDeco}
    H^{13}(\MM_{g,n})\otimes\CC = H^{12,1}(\MM_{g,n})\oplus H^{1,12}(\MM_{g,n})
\end{equation}
\[ H^{11}(\MM_{g,n})\otimes\CC = H^{11,0}(\MM_{g,n})\oplus H^{0,11}(\MM_{g,n}) \]
\[ H^{2}(\MM_{g,n})\otimes\CC = H^{1,1}(\MM_{g,n}) \]


In this paper we are interested in cohomology classes $\gamma_v\in H^k(\MM_{g,n})$ when viewed as decorations of a vertex of genus $g$ and valence $n$ in some graph $\Gamma$. Since it is cumbersome to bring along $\gamma$ whenever we want to reference a specific decorated graph $(\Gamma,\gamma)$, we will encode this datum in symbols drawn onto the vertex being decorated. The goal is to have graphical depictions that determine uniquely all relevant cohomology classes, so that just by the drawing it is possible to unambiguosly determine what is the decoration at each vertex. We will give these graphical depictions onto the one vertex graph $*_{g,n}$ of genus $g$ and with $n$ hairs, labeled by a set $N$; these depictions are understood to transfer onto the vertex being decorated of a general ambient graph.

To understand automorphisms of decorated graphs we will have to keep track of the $\ss_n$-action on each cohomolgy group induced by the permutation of $N$, which in some cases involves a sign representation.

In the following depictions, solid lines represent the minimum amount of edges \emph{necessary} for the considered class to exist, whereas dashed lines represent a \emph{potential} existance of edges.


\subsection{The case $k=0$, $g=0$, $n\geq 3$}

We have $H^{0,0}(\MM_{0,n})=\CC$ and thus we can draw weight $0$ vertices without any graphical depiction.
\[
\begin{tikzpicture}
    \node[int] (v) at (0,0) {};
    \draw (v) edge[dashed] +(.4,0)  edge +(-.4,.4) edge[dashed] +(.4,.4);
    \draw (v) edge +(-.4,0) edge +(-.4,-.4)  edge[dashed] +(.4,-.4);
\end{tikzpicture}
:= (*_{0,n},1)
\]

\subsection{The case $k=(1,1)$, $g=1$,$n=1$}
$H^{1,1}(\MM_{1,1})$ is one dimensional, spanned by a class that we call $\delta_{irr}$. Since this is the only case where a non special vertex might have genus $1$, we will introduce a symbolic loop with a crossed edge and draw the node black as if it were a genus $0$ vertex. As $n=1$, there is no $\ss_n$-action on $\delta_{irr}$ to talk about.
\[
\begin{tikzpicture}
    \node[int] (v) at (0,0) {};
    \draw (v) edge +(0,-.3) edge[loop, crossed, distance=0.8cm] (v);
\end{tikzpicture}
:= (*_{1,1},\delta_{irr})
\]


\subsection{The case $k=(1,1)$, $g=0$} For this case we refer to \cite[Section 3]{PayneWillwacher21}.
In $g=0$, $H^{1,1}(\MM_{0,n})$ is non zero only for $n\geq 4$; so we operate under this assumption.
The group is generated by classes $\psi_i$ for every $1\leq i\leq n$ and $\delta\{\substack{A \\ A'}\}$ for every partition $A\sqcup A'=N$ with $|A|,|A'|\geq 2$. To depict $\delta\{\substack{A \\ A'}\}$ we split symbolically the vertex in two parts connected by a crossed edge and draw on one side the subset of hairs $A$ and on the other $A'$. The notation $\{\substack{A \\ A'}\}$ is chosen to express the fact that swapping $A$ and $A'$ doesn't change the class, which graphically means it doesn't matter on which sides the two sets of hairs are chosen to be drawn. The $\ss_n$-action is given by $\sigma\,\psi_i = \psi_{\sigma i}$ and $\sigma\,\delta\{\substack{A \\ A'}\}=\delta\{\substack{\sigma A \\ \sigma A'}\}$.

\[
    \begin{tikzpicture}
        \node[int] (v) at (0,0) {};
        \draw (v) edge[dashed] +(.6,0)  edge +(-.6,.5) edge[->-] +(.6,.5);
        \draw (v) edge[dashed] +(-.6,0) edge +(-.6,-.5)  edge +(.6,-.5);
        \node (x) at (.8,.5) {\small $i$};
    \end{tikzpicture}
    := (*_{0,n},\psi_i)
    \hspace{1cm}
    \begin{tikzpicture}
        \node[int] (v) at (0,0) {};
        \node[int] (w) at (0.5,0) {};
        \draw (v) edge +(-.5,-.5) edge +(-.5,.5) edge[dashed] +(-.5,.2) edge[dashed] +(-.2,.5) edge[dashed] +(-.5,-.2) edge[dashed] +(-.2,-.5) edge[crossed] (w)
        (w) edge +(.5,-.5) edge +(.5,.5) edge[dashed] +(.5,.2) edge[dashed] +(.5,-.2) edge[dashed] +(.2,.5) edge[dashed] +(.2,-.5);
        \draw [decorate,decoration={brace,amplitude=5pt,mirror}]
        (-.7,.5) -- (-.7,-.5) node[midway,xshift=-1em]{$A$};
        \draw [decorate,decoration={brace,amplitude=5pt}]
        (1.1,.5) -- (1.1,-.5) node[midway,xshift=1em]{$A'$};
    \end{tikzpicture}
    := (*_{0,n},\delta\{\substack{A \\ A'}\})
\]
There are two equivalent families of relations between these generators. For any three pairwise distinct $i,x,y\in N$, or for any $i\neq j\in N$, it holds: 
\begin{equation} \label{equ:Relations2}
    \psi_i = \sum_{\substack{A\sqcup A'=N,|A|,|A'|\geq 2 \\ i\in A,x,y\in A'}} \delta\{\substack{A \\ A'}\}  \hspace{2cm}  \psi_i+\psi_j = \sum_{\substack{A\sqcup A'=N,|A|,|A'|\geq 2 \\ i\in A,j\in A'}} \delta\{\substack{A \\ A'}\}.
\end{equation}
So the $\psi_i$ classes are actually superfluous in the case $g=0$, but algebraically they can be more convenient to work with. The dimension of $H^{1,1}(\MM_{0,n})$ turns out to be $2^{n-1}- \binom{n}{2} -1$, in particular when $n=4$ every single class forms a basis.

\subsection{The case $k=(11,0)$, $g=1$} This case is studied in \cite[Section 2]{CLP}. $H^{11,0}(\MM_{1,n})$ is non zero only for $n\geq 11$; so we operate under this assumption. The group is generated by classes $\omega_B$ for every \emph{alternatingly ordered} $B\subseteq N$ with $|B|=11$. This means that the underlying set determines $\omega_B$ up to sign, and if we choose a canonical labeling $N=\{1,...,n\}$ we can stipulate that every subset comes equipped with the increasing ordering. We draw arrows onto the hairs contained in $B$ to depict the $\omega_B$ decoration.
\[
\begin{tikzpicture}
    \node[ext] (v) at (0,0) {};
    \node at (-.62, .1) {$\scriptscriptstyle \vdots$};
    \draw (v) edge[->-] +(-.6,-.5) edge[->-] +(-.6,-.3) edge[->-] +(-.3,-.5)
    edge[->-] +(-.6,.5) edge[->-] +(-.6,.3) edge[->-] +(-.3,.5) 
    edge[dashed] +(.6,.5) edge[dashed] +(.6,+.3)  edge[dashed] +(.3,+.5)
    edge[dashed] +(.6,-.5) edge[dashed] +(.6,-.3) edge[dashed] +(.3,-.5) ;
    \draw [decorate,decoration={brace,amplitude=5pt,mirror}]
  (-.8,.5) -- (-.8,-.5) node[midway,xshift=-1em]{$B$};
    \draw [decorate,decoration={brace,amplitude=5pt}]
    (.7,.5) -- (.7,-.5) node[midway,xshift=1em]{$B^c$};
\end{tikzpicture}
:= (*_{1,n},\omega_B)
\]

The only relations are amongst the classes $\omega_B$ whose set $B$ is contained in the same size $12$ subset. Namely, if we choose a canonical labeling on $N$, then for every $E=\{e_1,...,e_{12}\}\subseteq N$ with $e_i$ increasing there is the relation
\begin{equation} \label{equ:Relations11}
    \sum_{i=1}^{12} (-1)^i \omega_{E\setminus e_i} =0
\end{equation}
Therefore, choosing a distinguished hair $e\in N$ (for example $e=1$), the classes $\omega_B$ with $e\in B$ form a basis of $H^{11,0}(\MM_{1,n})$.

The $\ss_n$-action on $\omega_B$ is given by $\sigma\,\omega_B=\omega_{\sigma B}$, which is equal to $\sgn \sigma \,\omega_B$ if $\sigma$ preserves $B$ setwise.


\subsection{The case $k=(12,1)$, $g=1$} This case is studied in \cite[Section 4.2]{CLPW}.
$H^{12,1}(\MM_{1,n})$ is non zero only for $n\geq 12$; so we operate under this assumption. The group is generated by classes $Z_{B\subseteq A}$ for every subset $A\subseteq N$ with $|A^c|\geq 2$ and \emph{alternatingly ordered} subset $B\subseteq A$ with $|B|=10$; the underlying set $B$ determines $Z_{B\subseteq A}$ up to sign. We draw arrows onto the hairs contained in $B$, and we split symbolically the vertex in a genus $1$ vertex, where we attach the hairs in $A$, and a genus $0$ vertex, where we attach the hairs in $A^c$.

\[
\begin{tikzpicture}
    \node[ext] (v) at (0,0) {};
    \node at (-.62, .1) {$\scriptscriptstyle \vdots$};
    \node[int] (w) at (.8, 0) {};
    \draw (v) edge[->-] +(-.6,-.5) edge[->-] +(-.6,-.3) edge[->-] +(-.3,-.5)
    edge[->-] +(-.6,.5) edge[->-] +(-.6,.3) edge[->-] +(-.3,.5) 
    edge[dashed] +(.6,.5) edge[dashed] +(.6,+.3)  edge[dashed] +(.3,+.5)
    edge[dashed] +(.6,-.5) edge[dashed] +(.6,-.3) edge[dashed] +(.3,-.5) edge[crossed] (w);
    \draw (w) edge +(.6,.5) edge[dashed] +(.6,+.3)  edge[dashed] +(.3,+.5)
    edge +(.6,-.5) edge[dashed] +(.6,-.3) edge[dashed] +(.3,-.5);
    \draw [decorate,decoration={brace,amplitude=5pt,mirror}]
    (-.8,.5) -- (-.8,-.5) node[midway,xshift=-1em]{$B$};
    \draw [decorate,decoration={brace,amplitude=5pt}]
    (1.5,.5) -- (1.5,-.5) node[midway,xshift=1em]{$A^c$};
    \draw [decorate,decoration={brace,mirror,amplitude=5pt}]
    (-.7,-.6) -- (.7,-.6) node[midway,yshift=-0.8em]{\small $A$};
\end{tikzpicture}
:= (*_{1,n},Z_{B\subseteq A})
\]

The only relations are amongst the classes $Z_{B\subseteq A}$ having $|A^c|=2$ and same set $B\sqcup A^c$. Namely, if we choose a canonical labeling on $N$, then for every $E=\{e_1,...,e_{12}\}\subseteq N$ with $e_i$ increasing and every $1\leq i<j<k\leq 12$ we have the relation
\begin{equation} \label{equ:Relations13}
    (-1)^{i+j}Z_{E\setminus e_i,e_j \subseteq N\setminus e_i,e_j} - (-1)^{i+k}Z_{E\setminus e_i,e_k\subseteq N\setminus e_i,e_k} + (-1)^{j+k}Z_{E\setminus e_j,e_k\subseteq N\setminus e_j,e_k} = 0
\end{equation}
For this subset $E\subseteq N$, choosing a distinguished element $e\in E$ (for example $e=e_1$), the subspace $PB_E$ spanned by the classes $Z_{B\subseteq A}$ with $B\sqcup A^c=E$ has basis the ones with $e\in A^c$, of which there are $11$. So $H^{12,1}(\MM_{1,n})$ decomposes into a direct sum $PB_3 \oplus \bigoplus_{|E|=12} PB_E$, where $PB_3$ has basis the classes with $|A^c|\geq 3$.
The $\ss_n$-action on $Z_{B\subseteq A}$ is given by $\sigma\,Z_{B\subseteq A}=Z_{\sigma B\subseteq \sigma A}$, which is equal to $\sgn \sigma \,Z_{B\subseteq \sigma A}$ if $\sigma$ preserves $B$ setwise.





\subsection{Action of the differential on cohomology classes} \label{sec:1differential}
In this section we describe the action of the differential operator that splits one-vertex decorated graphs.
A splitting of a decorated graph $(*_{g,n},\gamma)$ is of the form $(*'-*'',\gamma'\otimes\gamma'')$, where $*'-*''$ is the connection of two vertices $*'_{g',n'},*''_{g'',n''}$ with $g=g'+g''$, $n=n'+n''+2$, and $\gamma',\gamma''$ are their respective decorations.
The hairs of $*'$ and $*''$ form a partition of $N$. If $q'$ and $q''$ are the two half-edges connecting $*'$ and $*''$, then $\gamma'$ is obtained from $\gamma$ by pullback along the map $\MM_{g',n'+1}\rightarrow \MM_{g,n}$ determined by the subset of hairs ending up on $*'$. For the computation of the pullbacks of cohomology classes we refer to \cite{PayneWillwacher21},\cite{CLP} and \cite{CLPW} for the weight $2$, weight $11$ and weight $13$ cases respectively, in this paper we limit ourselves to translating those computations in graphical form.\\

The image under the differential is given by summing over these two-vertex decorated graphs for all possible splittings. In the case of interest, we always have either $g'=g''=g=0$ or $g'=g=1,g''=0$, so the only determining datum of the splitting is the partition of the hairs of $*_{g,n}$.
%In the terms where the alternatigly ordered subsets $B$ are modified in \ref{equ:Diff11} and \ref{equ:Diff12}, the half-edge $q$ is understood to take $\tilde{b}$s place in the ordering of $B$. In the two terms where a weight $11$ decoration is created out of a weight $13$ one, the half-edge $q$ is understood to be the last in the ordering of $B\sqcup q$.

\begin{equation} \label{equ:Diff1}
\begin{tikzpicture}
    \node[int] (v) at (0,0) {};
    \draw (v) edge[dashed] +(.4,0)  edge +(-.4,.4) edge +(.4,.4);
    \draw (v) edge[dashed] +(-.4,0) edge +(-.4,-.4)  edge +(.4,-.4);
\end{tikzpicture}
\hspace{.3in}
\xlongrightarrow{d}
\hspace{.3in}
\sum_{\substack{S\sqcup S' = N \\ |S|,|S'|\geq 2}}
\begin{tikzpicture}
    \node[int] (v) at (0,0) {};
    \node[int] (w) at (.6,0) {};
    \draw (v) edge[dashed] +(-.4,0)  edge +(-.4,.4) edge +(-.4,-.4) edge (w);
    \draw (w) edge[dashed] +(.4,0) edge +(.4,.4)  edge +(.4,-.4);
    \draw [decorate,decoration={brace,amplitude=5pt,mirror}]
    (-.5,.5) -- (-.5,-.5) node[midway,xshift=-1em]{$S$};
    \draw [decorate,decoration={brace,amplitude=5pt}]
    (1.1,.5) -- (1.1,-.5) node[midway,xshift=1em]{$S'$};
\end{tikzpicture}
\end{equation}


\begin{equation} \label{equ:Diff2irr}
    \begin{tikzpicture}
        \node[int] (v) at (0,0) {};
        \draw (v) edge +(0,-.3) edge[loop, crossed, distance=0.8cm] (v);
    \end{tikzpicture}
\hspace{.1in}
\xlongrightarrow{d}
\hspace{.1in}
0 \hspace{.3in} \text{because the valence of $*_{1,1}$ is less than $2$}
\end{equation}


\begin{multline} \label{equ:Diff2}
\text{For any choice of $x,y\in A$ and $x',y'\in A'$, the image can be expressed as follows:}\\
\begin{tikzpicture}
    \node[int] (v) at (0,0) {};
    \node[int] (w) at (0.5,0) {};
    \draw (v) edge +(-.5,-.5) edge +(-.5,.5) edge[dashed] +(-.5,.2) edge[dashed] +(-.2,.5) edge[dashed] +(-.5,-.2) edge[dashed] +(-.2,-.5) edge[crossed] (w)
    (w) edge +(.5,-.5) edge +(.5,.5) edge[dashed] +(.5,.2) edge[dashed] +(.5,-.2) edge[dashed] +(.2,.5) edge[dashed] +(.2,-.5);
    \draw [decorate,decoration={brace,amplitude=5pt,mirror}]
    (-.7,.5) -- (-.7,-.5) node[midway,xshift=-1em]{$A$};
    \draw [decorate,decoration={brace,amplitude=5pt}]
    (1.1,.5) -- (1.1,-.5) node[midway,xshift=1em]{$A'$};
\end{tikzpicture}
\hspace{.05in}
\xlongrightarrow{d}
\hspace{.05in}
-\sum_{x,y\in\tilde{A}\subset A}
\begin{tikzpicture}
    \node[int] (v) at (0,0) {};
    \node[int] (w) at (0.5,0) {};
    \node[int] (z) at (1,0) {};
    \node[] (x) at (-.6,.5) {\tiny $x$};
    \node[] (y) at (-.6,.2) {\tiny $y$};
    \draw (v)  edge +(-.5,.5) edge +(-.5,.2) edge[dashed] +(-.2,.5)  edge[crossed] (w)
    (w) edge (z) edge +(-.5,-.5) edge[dashed] +(-.5,-.3) edge[dashed] +(-.2,-.5);
    \draw (z) edge +(.5,-.5) edge +(.5,.5) edge[dashed] +(.5,.2) edge[dashed] +(.5,-.2) edge[dashed] +(.2,.5) edge[dashed] +(.2,-.5);
    \draw [decorate,decoration={brace,amplitude=5pt,mirror}]
    (-.8,.6) -- (-.8,0) node[midway,xshift=-1em]{\tiny $\tilde{A}$};
    \draw [decorate,decoration={brace,amplitude=5pt,mirror}]
    (-.2,0) -- (-.2,-.5) node[midway,xshift=-1.5em]{\tiny $A\setminus\tilde{A}$};
    \draw [decorate,decoration={brace,amplitude=5pt}]
    (1.6,.5) -- (1.6,-.5) node[midway,xshift=1em]{$A'$};
\end{tikzpicture}
-\sum_{x',y'\in\tilde{A}\subset A'}
\begin{tikzpicture}
    \node[int] (z) at (-1,0) {};
    \node[int] (v) at (-0.5,0) {};
    \node[int] (w) at (0,0) {};
    \node[] (x) at (.7,.5) {\tiny $x'$};
    \node[] (y) at (.7,.2) {\tiny $y'$};
    \draw (z) edge +(-.5,-.5) edge +(-.5,.5) edge[dashed] +(-.5,.2) edge[dashed] +(-.2,.5) edge[dashed] +(-.5,-.2) edge[dashed] +(-.2,-.5) edge (v);
    \draw (v) edge[crossed] (w) edge +(.5,-.5) edge[dashed] +(.5,-.3) edge[dashed] +(.2,-.5);
    \draw (w) edge +(.5,.5) edge +(.5,.2) edge[dashed] +(.2,.5) ;
    \draw [decorate,decoration={brace,amplitude=5pt,mirror}]
    (-1.7,.5) -- (-1.7,-.5) node[midway,xshift=-1em]{$A$};
    \draw [decorate,decoration={brace,amplitude=5pt}]
    (.9,.6) -- (.9,0) node[midway,xshift=1em]{\tiny $\tilde{A}$};
    \draw [decorate,decoration={brace,amplitude=5pt}]
    (.2,0) -- (.2,-.5) node[midway,xshift=1.5em]{\tiny $A'\setminus\tilde{A}$};
\end{tikzpicture}\\
+\sum_{\substack{S\subset A' \\ |S|\geq 2}}
\begin{tikzpicture}
    \node[int] (z) at (-1,0) {};
    \node[int] (v) at (-0.5,0) {};
    \node[int] (w) at (0,0) {};
    \draw (z) edge +(-.5,-.5) edge +(-.5,.5) edge[dashed] +(-.5,.2) edge[dashed] +(-.2,.5) edge[dashed] +(-.5,-.2) edge[dashed] +(-.2,-.5) edge[crossed] (v);
    \draw (v) edge (w) edge +(.5,-.5) edge[dashed] +(.5,-.3) edge[dashed] +(.2,-.5);
    \draw (w) edge +(.5,.5) edge +(.5,.2) edge[dashed] +(.2,.5) ;
    \draw [decorate,decoration={brace,amplitude=5pt,mirror}]
    (-1.7,.5) -- (-1.7,-.5) node[midway,xshift=-1em]{$A$};
    \draw [decorate,decoration={brace,amplitude=5pt}]
    (.6,.6) -- (.6,0) node[midway,xshift=1em]{\tiny $S$};
    \draw [decorate,decoration={brace,amplitude=5pt}]
    (.2,0) -- (.2,-.5) node[midway,xshift=1.5em]{\tiny $A'\setminus S$};
\end{tikzpicture}
+\sum_{\substack{S\subset A \\ |S|\geq 2}}
\begin{tikzpicture}
    \node[int] (v) at (0,0) {};
    \node[int] (w) at (0.5,0) {};
    \node[int] (z) at (1,0) {};
    \draw (v)  edge +(-.5,.5) edge +(-.5,.2) edge[dashed] +(-.2,.5)  edge[crossed] (w)
    (w) edge (z) edge +(-.5,-.5) edge[dashed] +(-.5,-.3) edge[dashed] +(-.2,-.5);
    \draw (z) edge +(.5,-.5) edge +(.5,.5) edge[dashed] +(.5,.2) edge[dashed] +(.5,-.2) edge[dashed] +(.2,.5) edge[dashed] +(.2,-.5);
    \draw [decorate,decoration={brace,amplitude=5pt,mirror}]
    (-.6,.6) -- (-.6,0) node[midway,xshift=-1em]{\tiny $S$};
    \draw [decorate,decoration={brace,amplitude=5pt,mirror}]
    (-.2,0) -- (-.2,-.5) node[midway,xshift=-1.5em]{\tiny $A\setminus S$};
    \draw [decorate,decoration={brace,amplitude=5pt}]
    (1.6,.5) -- (1.6,-.5) node[midway,xshift=1em]{$A'$};
\end{tikzpicture}
\end{multline}


In the second term of \ref{equ:Diff11}, the weight $11$ decoration $\omega_B$ becomes $\omega_{B\setminus\tilde{b}\sqcup q}$, where $q$ is the newly added half-edge to the genus $1$ vertex and takes the place of $\tilde{b}$ in the ordering of $B$.
\begin{equation} \label{equ:Diff11}
    \begin{tikzpicture}
        \node[ext] (v) at (0,0) {};
        \node at (-.62, .1) {$\scriptscriptstyle \vdots$};
        \draw (v) edge[->-] +(-.6,-.5) edge[->-] +(-.6,-.3) edge[->-] +(-.3,-.5)
        edge[->-] +(-.6,.5) edge[->-] +(-.6,.3) edge[->-] +(-.3,.5) 
        edge[dashed] +(.6,.5) edge[dashed] +(.6,+.3)  edge[dashed] +(.3,+.5)
        edge[dashed] +(.6,-.5) edge[dashed] +(.6,-.3) edge[dashed] +(.3,-.5) ;
        \draw [decorate,decoration={brace,amplitude=5pt,mirror}]
      (-.8,.5) -- (-.8,-.5) node[midway,xshift=-1em]{$B$};
        \draw [decorate,decoration={brace,amplitude=5pt}]
        (.7,.5) -- (.7,-.5) node[midway,xshift=1em]{$B^c$};
    \end{tikzpicture}
    \hspace{.05in}
    \xlongrightarrow{d}
    \hspace{.05in}
    \sum_{\substack{S\subseteq B^c \\ |S|\geq 2}}
    \begin{tikzpicture}
        \node[ext] (v) at (0,0) {};
        \node at (-.62, .1) {$\scriptscriptstyle \vdots$};
        \node[int] (z) at (.5,0) {};
        \draw (v) edge[->-] +(-.6,-.5) edge[->-] +(-.6,-.3) edge[->-] +(-.3,-.5)
        edge[->-] +(-.6,.5) edge[->-] +(-.6,.3) edge[->-] +(-.3,.5) 
        edge[dashed] +(.6,-.5) edge[dashed] +(.6,-.3) edge[dashed] +(.3,-.5) edge (z);
        \draw (z) edge +(.6,.5) edge +(.6,+.3)  edge[dashed] +(.3,+.5);
        \draw [decorate,decoration={brace,amplitude=5pt,mirror}]
      (-.8,.5) -- (-.8,-.5) node[midway,xshift=-1em]{$B$};
        \draw [decorate,decoration={brace,amplitude=5pt}]
        (1.2,.6) -- (1.2,0) node[midway,xshift=1em]{\tiny $S$};
        \draw [decorate,decoration={brace,amplitude=5pt}]
        (.7,0) -- (.7,-.5) node[midway,xshift=1.5em]{\tiny $B^c\setminus S$};
    \end{tikzpicture}
    +\sum_{\substack{\varnothing\neq S\subseteq B^c \\ \tilde{b}\in B}}
    \begin{tikzpicture}
        \node[ext] (v) at (0,0) {};
        \node at (-.62, .1) {$\scriptscriptstyle \vdots$};
        \node[int] (z) at (.7,0) {};
        \node (b) at (1.4,.6) {\tiny $\tilde{b}$};
        \draw (v) edge[->-] +(-.6,-.3) edge[->-] +(-.6,-.5) edge[->-] +(-.3,-.5)
        edge[->-] +(-.6,.3) edge[->-] +(-.3,.5) 
        edge[dashed] +(.6,-.5) edge[dashed] +(.6,-.3) edge[dashed] +(.3,-.5) edge[->-] (z);
        \draw (z) edge +(.6,.5) edge +(.6,+.3)  edge[dashed] +(.3,+.5);
        \draw [decorate,decoration={brace,amplitude=5pt,mirror}]
      (-.7,.5) -- (-.7,-.5) node[midway,xshift=-1.3em]{\tiny $B\setminus\tilde{b}$};
        \draw [decorate,decoration={brace,amplitude=5pt}]
        (1.4,.35) -- (1.4,0) node[midway,xshift=1em]{\tiny $S$};
        \draw [decorate,decoration={brace,amplitude=5pt}]
        (.9,0) -- (.9,-.5) node[midway,xshift=1.5em]{\tiny $B^c\setminus S$};
    \end{tikzpicture}
\end{equation}


For any fixed choice of $x,y\in A^c$, the image can be expressed as in \ref{equ:Diff12}. In the two terms that create weight $11$ and $2$ vertices, the newly created weight $11$ decoration $\omega_{B\sqcup p}$, where $p$ is the half-edge at the genus $1$ vertex, is understood to have the ordering inherited from $B$ with $p$ appended at the end.
\begin{multline}\label{equ:Diff12}  
\begin{tikzpicture}
    \node[ext] (v) at (0,0) {};
    \node at (-.62, .1) {$\scriptscriptstyle \vdots$};
    \node[int] (w) at (.8, 0) {};
    \draw (v) edge[->-] +(-.6,-.5) edge[->-] +(-.6,-.3) edge[->-] +(-.3,-.5)
    edge[->-] +(-.6,.5) edge[->-] +(-.6,.3) edge[->-] +(-.3,.5) 
    edge[dashed] +(.6,.5) edge[dashed] +(.6,+.3)  edge[dashed] +(.3,+.5)
    edge[dashed] +(.6,-.5) edge[dashed] +(.6,-.3) edge[dashed] +(.3,-.5) edge[crossed] (w);
    \draw (w) edge +(.6,.5) edge[dashed] +(.6,+.3)  edge[dashed] +(.3,+.5)
    edge +(.6,-.5) edge[dashed] +(.6,-.3) edge[dashed] +(.3,-.5);
    \draw [decorate,decoration={brace,amplitude=5pt,mirror}]
    (-.8,.5) -- (-.8,-.5) node[midway,xshift=-1em]{$B$};
    \draw [decorate,decoration={brace,amplitude=5pt}]
    (1.5,.5) -- (1.5,-.5) node[midway,xshift=1em]{$A^c$};
\end{tikzpicture}
\hspace{.05in}
\xlongrightarrow{d}
\hspace{.05in}
\sum_{\substack{S\subset A^c \\ |S|\geq 2}}
\begin{tikzpicture}
    \node[ext] (v) at (0,0) {};
    \node at (-.62, .1) {$\scriptscriptstyle \vdots$};
    \node[int] (w) at (.8, 0) {};
    \node[int] (z) at (1.3,0) {};
    \draw (v) edge[->-] +(-.6,-.5) edge[->-] +(-.6,-.3) edge[->-] +(-.3,-.5)
    edge[->-] +(-.6,.5) edge[->-] +(-.6,.3) edge[->-] +(-.3,.5) 
    edge[dashed] +(.6,.5) edge[dashed] +(.6,+.3)  edge[dashed] +(.3,+.5)
    edge[dashed] +(.6,-.5) edge[dashed] +(.6,-.3) edge[dashed] +(.3,-.5) edge[crossed] (w);
    \draw (w) edge +(.6,-.5) edge[dashed] +(.6,-.3) edge[dashed] +(.3,-.5) edge (z);
    \draw (z) edge +(.6,.5) edge +(.6,+.3)  edge[dashed] +(.3,+.5);
    \draw [decorate,decoration={brace,amplitude=5pt,mirror}]
    (-.8,.5) -- (-.8,-.5) node[midway,xshift=-1em]{$B$};
    \draw [decorate,decoration={brace,amplitude=5pt}]
    (2,.6) -- (2,0) node[midway,xshift=1em]{\tiny $S$};
    \draw [decorate,decoration={brace,amplitude=5pt}]
    (1.5,0) -- (1.5,-.5) node[midway,xshift=1.5em]{\tiny $A^c\setminus S$};
\end{tikzpicture}
+\sum_{\substack{\tilde{S}\subseteq A\setminus B \\ |\tilde{S}|\geq 2}}
\begin{tikzpicture}
    \node[ext] (v) at (0,0) {};
    \node at (-.62, .1) {$\scriptscriptstyle \vdots$};
    \node[int] (w) at (.8, 0) {};
    \node[int] (z) at (.2,.5) {};
    \draw (v) edge (z) edge[->-] +(-.6,-.5) edge[->-] +(-.6,-.3) edge[->-] +(-.3,-.5)
    edge[->-] +(-.6,.5) edge[->-] +(-.6,.3) edge[->-] +(-.3,.5) 
    edge[dashed] +(.6,-.5) edge[dashed] +(.6,-.3) edge[dashed] +(.3,-.5) edge[crossed] (w);
    \draw (z) edge +(.6,.5) edge +(.6,+.3)  edge[dashed] +(.3,+.5);
    \draw (w) edge +(.6,.5) edge[dashed] +(.6,+.3)  edge[dashed] +(.3,+.5)
    edge +(.6,-.5) edge[dashed] +(.6,-.3) edge[dashed] +(.3,-.5);
    \draw [decorate,decoration={brace,amplitude=5pt,mirror}]
    (-.8,.5) -- (-.8,-.5) node[midway,xshift=-1em]{$B$};
    \draw [decorate,decoration={brace,amplitude=5pt}]
    (.9,1.1) -- (.9,.6) node[midway,xshift=1em]{\tiny $\tilde{S}$};
    \draw [decorate,decoration={brace,amplitude=5pt}]
    (1.5,.5) -- (1.5,-.5) node[midway,xshift=1em]{$A^c$};
\end{tikzpicture}\\
+\sum_{\substack{\varnothing\neq\tilde{S}\subseteq A\setminus B \\ \tilde{b}\in B}}
\begin{tikzpicture}
    \node[ext] (v) at (0,0) {};
    \node at (-.62, .1) {$\scriptscriptstyle \vdots$};
    \node[int] (w) at (.8, 0) {};
    \node[int] (z) at (.2,.5) {};
    \node (b) at (-.3,.8) {\tiny $\tilde{b}$};
    \draw (v) edge[->-] (z) edge[->-] +(-.6,-.5) edge[->-] +(-.6,-.3) edge[->-] +(-.3,-.5)
    edge[->-] +(-.6,.5) edge[->-] +(-.6,.3)
    edge[dashed] +(.6,-.5) edge[dashed] +(.6,-.3) edge[dashed] +(.3,-.5) edge[crossed] (w);
    \draw (z) edge +(.6,.5) edge[dashed] +(.6,+.3)  edge[dashed] +(.3,+.5) edge +(-.4,.5);
    \draw (w) edge +(.6,.5) edge[dashed] +(.6,+.3)  edge[dashed] +(.3,+.5)
    edge +(.6,-.5) edge[dashed] +(.6,-.3) edge[dashed] +(.3,-.5);
    \draw [decorate,decoration={brace,amplitude=5pt,mirror}]
    (-.7,.5) -- (-.7,-.5) node[midway,xshift=-1.3em]{\tiny$B\setminus\tilde{b}$};
    \draw [decorate,decoration={brace,amplitude=5pt}]
    (.9,1.1) -- (.9,.6) node[midway,xshift=1em]{\tiny $\tilde{S}$};
    \draw [decorate,decoration={brace,amplitude=5pt}]
    (1.5,.5) -- (1.5,-.5) node[midway,xshift=1em]{$A^c$};
\end{tikzpicture}
+\sum_{\varnothing\neq\tilde{S}\subseteq A\setminus B}
\begin{tikzpicture}
    \node[ext] (v) at (0,0) {};
    \node at (-.62, .1) {$\scriptscriptstyle \vdots$};
    \node[int] (w) at (.7, 0) {};
    \node[int] (z) at (1.4,0) {};
    \draw (v) edge[->-] +(-.6,-.5) edge[->-] +(-.6,-.3) edge[->-] +(-.3,-.5)
    edge[->-] +(-.6,.5) edge[->-] +(-.6,.3) edge[->-] +(-.3,.5) 
    edge[dashed] +(.6,-.5) edge[dashed] +(.6,-.3) edge[dashed] +(.3,-.5) edge[->-] (w);
    \draw (w)  edge[crossed] (z) edge +(.4,.5) edge[dashed] +(.6,+.4)  edge[dashed] +(.2,+.5);
    \draw (z) edge +(.6,.5) edge[dashed] +(.6,+.3)  edge[dashed] +(.3,+.5)
    edge +(.6,-.5) edge[dashed] +(.6,-.3) edge[dashed] +(.3,-.5);
    \draw [decorate,decoration={brace,amplitude=5pt,mirror}]
    (-.8,.5) -- (-.8,-.5) node[midway,xshift=-1em]{$B$};
    \draw [decorate,decoration={brace,amplitude=5pt}]
    (2.1,.5) -- (2.1,-.5) node[midway,xshift=1em]{$A^c$};
    \draw [decorate,decoration={brace,amplitude=5pt}]
    (.8,.5) -- (1.3,.5) node[midway,yshift=1em]{\tiny $\tilde{S}$};
\end{tikzpicture}\\
-\sum_{\varnothing\neq\tilde{S}\subseteq A\setminus B}
\begin{tikzpicture}
    \node[ext] (v) at (0,0) {};
    \node at (-.62, .1) {$\scriptscriptstyle \vdots$};
    \node[int] (w) at (.7, 0) {};
    \node[int] (z) at (1.4,0) {};
    \draw (v) edge[->-] +(-.6,-.5) edge[->-] +(-.6,-.3) edge[->-] +(-.3,-.5)
    edge[->-] +(-.6,.5) edge[->-] +(-.6,.3) edge[->-] +(-.3,.5) 
    edge[dashed] +(.6,-.5) edge[dashed] +(.6,-.3) edge[dashed] +(.3,-.5) edge[crossed] (w);
    \draw (w)  edge (z) edge +(.4,.5) edge[dashed] +(.6,+.4)  edge[dashed] +(.2,+.5);
    \draw (z) edge +(.6,.5) edge[dashed] +(.6,+.3)  edge[dashed] +(.3,+.5)
    edge +(.6,-.5) edge[dashed] +(.6,-.3) edge[dashed] +(.3,-.5);
    \draw [decorate,decoration={brace,amplitude=5pt,mirror}]
    (-.8,.5) -- (-.8,-.5) node[midway,xshift=-1em]{$B$};
    \draw [decorate,decoration={brace,amplitude=5pt}]
    (2.1,.5) -- (2.1,-.5) node[midway,xshift=1em]{$A^c$};
    \draw [decorate,decoration={brace,amplitude=5pt}]
    (.8,.5) -- (1.3,.5) node[midway,yshift=1em]{\tiny $\tilde{S}$};
\end{tikzpicture}
-\sum_{x,y\in S\subset A^c}
\begin{tikzpicture}
    \node[ext] (v) at (0,0) {};
    \node at (-.62, .1) {$\scriptscriptstyle \vdots$};
    \node[int] (w) at (.7, 0) {};
    \node[int] (z) at (1.3,0) {};
    \node (x) at (2,.5) {\tiny $x$};
    \node (y) at (2,.2) {\tiny $y$};
    \draw (v) edge[->-] +(-.6,-.5) edge[->-] +(-.6,-.3) edge[->-] +(-.3,-.5)
    edge[->-] +(-.6,.5) edge[->-] +(-.6,.3) edge[->-] +(-.3,.5) 
    edge[dashed] +(.6,.5) edge[dashed] +(.6,+.3)  edge[dashed] +(.3,+.5)
    edge[dashed] +(.6,-.5) edge[dashed] +(.6,-.3) edge[dashed] +(.3,-.5) edge[->-] (w);
    \draw (w) edge +(.6,-.5) edge[dashed] +(.6,-.3) edge[dashed] +(.3,-.5) edge[crossed] (z);
    \draw (z) edge +(.6,.5) edge +(.6,+.3)  edge[dashed] +(.3,+.5);
    \draw [decorate,decoration={brace,amplitude=5pt,mirror}]
    (-.8,.5) -- (-.8,-.5) node[midway,xshift=-1em]{$B$};
    \draw [decorate,decoration={brace,amplitude=5pt}]
    (2.2,.6) -- (2.1,0) node[midway,xshift=1em]{\tiny $S$};
    \draw [decorate,decoration={brace,amplitude=5pt}]
    (1.4,-.1) -- (1.4,-.6) node[midway,xshift=1.5em]{\tiny $A^c\setminus S$};
\end{tikzpicture}
\end{multline}

